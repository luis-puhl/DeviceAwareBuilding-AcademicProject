\section{Aplicação Distribuidora - Gateway}
\label{sec:app-gw}

Para a construção do \emph{Gateway} foi feita a instalação e a configuração do
MQTT \emph{Broker} Mosquitto em um dos RPI3.

\begin{citacao}

	Eclipse Mosquitto™ é um distribuidor de mensagens de código aberto (EPL/EDL
	licenciado) que implementa o protocolo MQTT versões 3.1 e 3.1.1. O MQTT
	fornece um método leve de transmitir mensagens usando um modelo de
	publicação/inscrição. Isso o torna adequado para mensagens "Internet das
	Coisas", como com sensores de baixa potência ou dispositivos móveis, como
	telefones, computadores embutidos ou microcontroladores como o Arduino. \

	\citeonline{mosquitto} Tradução Nossa.
\end{citacao}


A instalação é realizada com o gerenciador de pacotes \emph{apt-get} padrão do
raspbian como mostrado na primeira linha do \autoref{code-mosquitto}.

Nas linhas 2 e 3, adiciona-se ao arquivo padrão de configuração padrão do
\emph{Mosquitto} a linha \texttt{``password\_file /etc/mosquitto/passwd''} que
indica o arquivo de senhas a ser utilizado para autenticação. Na linha 4, é
adicionado o usuário ``user'' ao arquivo, a senha é solicitada  nas
linhas 5 e 6.

\begin{lstlisting}[language=bash,caption={Instalação e configuração do Mosquitto},label=code-mosquitto]
pi@broker:~ $ sudo apt-get install mosquitto
pi@broker:~ $ sudo sh -c 'echo "password_file /etc/mosquitto/passwd"
	>> /etc/mosquitto/mosquitto.conf'
pi@broker:~ $ sudo mosquitto_passwd -c /etc/mosquitto/passwd user
Password:
Reenter password:
pi@broker:~ $
\end{lstlisting}

Feito isso, o acesso ao \emph{Broker} está limitado aos usuários e senhas
configurados neste processo. As demais configurações permaneceram sem alterações.

Na configuração do sensor, deve ser adicionado o endereço (nome ou IP), a porta
\texttt{1883} e o par usuário e senha para que o sensor acesse o \emph{Gateway} com
sucesso.

Para verificar o funcionamento do conjunto sensor e distribuidor (\emph{Gateway}), utiliza-se um
cliente \emph{MQTT}, como o aplicativo para a plataforma Android ``MQTT Dashboard''
\cite{mqttdash} ou os aplicativos para as plataformas Java ``mqtt-spy''
\cite{mqttspy} e Windows ``MQTT.fx'' \cite{mqttfx}.

Nas figuras \ref{fig-ad-publish}, \ref{fig-ad-home-ack} e \ref{fig-ad-admin-ack},
o aplicativo ``MQTT Dashboard'' é utilizado para verificar quais sensores
estão online com a mensagem \texttt{``echo''} publicada no tópico \texttt{``ADMIN''}. Na
\autoref{fig-ad-publish}, os botões enviam mensagens com simplicidade. Na
\autoref{fig-ad-home-ack}, é visível a lista de tópicos e a última mensagem
recebida em cada um deles. Na \autoref{fig-ad-admin-ack}, fica evidente a
resposta de confirmação (\texttt{ack}) de cada um dos sensores.

\begin{figure}[htb]
	\centering
	\begin{minipage}{0.49\textwidth}
		\centering
		\caption{\label{fig-ad-publish}MQTT Dashboard: Envio de mensagens}
		\includegraphics[width=1\textwidth]{052-gateway/mqtt/ad-publish.jpg}
		\legend{Fonte: Produzido pelo autor}
	\end{minipage}
	\hfill
	\begin{minipage}{0.49\textwidth}
		\centering
		\caption{\label{fig-ad-home-ack}MQTT Dashboard: Lista de inscrições}
		\includegraphics[width=1\textwidth]{052-gateway/mqtt/ad-home-ack.jpg}
		\legend{Fonte: Produzido pelo autor}
	\end{minipage}
\end{figure}

\begin{figure}[htb]
	\centering
	\begin{minipage}{0.49\textwidth}
		\centering
		\caption{\label{fig-ad-admin-ack}MQTT Dashboard: Lista de mensagens no tópico}
		\includegraphics[width=1\textwidth]{052-gateway/mqtt/ad-admin-ack.jpg}
		\legend{Fonte: Produzido pelo autor}
	\end{minipage}
	\hfill
	\begin{minipage}{0.49\textwidth}
		\centering
		\caption{\label{fig-ad-devices-list}MQTT Dashboard: Mensagem de relatório completo do sensor}
		\includegraphics[width=1\textwidth]{052-gateway/mqtt/ad-devices-list.jpg}
		\legend{Fonte: Produzido pelo autor}
	\end{minipage}
\end{figure}

\FloatBarrier

A aplicação ``MQTT.fx'', em especial a tela mostrada na
\autoref{fig-mqttfx-stats}, revela estatísticas do \emph{Broker}, como a versão,
tempo \emph{online}, número de clientes, mensagens e utilização de rede.

\begin{figure}[htb]
	\centering
	\caption{\label{fig-mqttfx-stats}MQTT.fx: Estatísticas do \emph{Broker}}
	\includegraphics[width=1\textwidth]{052-gateway/mqtt/mqttfx-stats.png}
	\legend{Fonte: Produzido pelo autor}
\end{figure}

Para demonstrar a utilização da aplicação ``mqtt-spy'', o comando de listagem
(mencionado na \autopageref{code-mqttjs-client}) é demonstrado na
\autoref{fig-mqtt-spy-list}.

\begin{figure}[htb]
	\centering
	\caption{\label{fig-mqtt-spy-list}mqtt-spy: Listagem de dispositivos}
	\includegraphics[width=1\textwidth]{052-gateway/mqtt/mqtt-spy-list.png}
	\legend{Fonte: Produzido pelo autor}
\end{figure}
