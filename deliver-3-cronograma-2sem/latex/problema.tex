\chapter{PROBLEMA}
\label{chap:PROBLEMA}

Tamanha quantidade de dispositivos conectados pouco acrescenta na vida diária se
humanos ou coisas não puderem simplesmente se encontrar, tanto em ambiente real
quanto virtual é necessário o contato entre as partes para a existência de uma
interação.

Mais ainda, para melhor funcionamento de aplicações como o uso de conteúdo
específico para cada usuário e situação é necessário coletar informações
contextuais. Para a maioria das aplicações, a informação contextual de maior
relevância é a localização física.

Destacamos a necessidade da criação desta informação através de sensores ativos
sempre que necessário para que o dispositivo tenha ciência deste contexto em
suas tomadas de decisão e para que outros (sistemas, pessoas e coisas) saibam a
localização de qualquer dispositivo ao qual têm interesse de interagir.

Um exemplo da necessidade de localização de dispositivos dentro de um prédio
seria um profissional saber onde está o dispositivo em seu local de trabalho,
seja ele um vendedor e seu tablet para demostrar um produto fora de estoque em
uma loja ou um médico e um desfibrilador.

\section{SOBRE SISTEMAS DE POSICIONAMENTO}
\label{sec:SOBRE SISTEMAS DE POSICIONAMENTO}

Sistemas de posicionamento (PS - \textit{Positioning System}) são geralmente
constituídos de um Ponto Origem Global escolhido (\textit{O}) e um conjuto não
vazio de Pontos de Referência (RP - \textit{Reference Point}) cuja localização
global em relação ao \textit{O} é conhecida com precisão maior ou igual a
oferecida pelo sistema.

Também faz parte do sistema o ponto móvel (MU - \textit{Mobile User}) sobre o
qual temos interesse em determinar a posição que é feita pelo PS encontrando uma
distância (com dimensão variável de acordo com o método utilizado para
adquirir a distância) relativa a um sub-conjunto de RPs. Feito isso, é possível utilizar
modelos matemáticos para, a partir das distâncias, encontrar uma posição do MU
em relação aos RPs e uma nova transformação é aplicada para encontrar a posição
relativa ao \textit{O}.

Uma das maneiras de classificar PSs é entre os de Auto Posicionamento e
Posicionamento Remoto. Os de Auto Posicionamento contém no MU todo aparato
necessário para medir a distância dos RPs e calcular a posição em relação a
\textit{O}. Já os de Posicionamento Remoto tem o mínimo necessário na MU e todo
o trabalho de cálculo de distância e posição global é feito nos RPs ou em uma
unidade coordenadora destes.

Para PSs eletrônicos baseados em radio-frequência (RF - \textit{Radio
Frequency}), geralmente, utilizam-se dois componentes básicos, Transmissores e
Receptores, os quais assume-se que ao menos um destes está no RP e ao menos um
outro no MU. Para calcular a distância entre MU e RP, utiliza-se as propriedades
da comunicação por RF como tempo de chegada (TOA - \textit{Time Of Arrival}),
diferencial de tempo de chegada (TDOA - \textit{Time Difference Of Arrival}) e
ângulo de chegada de sinal (AOA - \textit{Angle Of Arrival}).

Para maior precisão, é comum a utilização de múltiplas RPs geralmente com o
número mínimo igual ao número de dimensões espaciais que deseja-se calcular.
Notamos que para sistemas distribuídos a sincronização de relógios é um problema
clássico então o tempo conta como dimensão.

Os sistemas classificados como ``Sistema de Navegação Global por Satélite''
(GNSS - \textit{Global Navigation Satellite System}), como o tradicional
Estadunidense Sistema de Posicionamento Global (GPS - \textit{Global Positioning
System}), utilizam a técnica em que o dispositivo móvel contém o receptor e os
transmissores são fixos em satélites na órbita terrestre \cite{Djuknic2001}.
Devido a posição e número de satélites, o GPS e seus correlatos estão sempre
presentes do ponto de vista de um observador da superfície terrestre, sendo para
este tipo de usuário um sistema ubíquo.

Entretanto, a força do sinal GNSS não é suficiente para penetrar a maioria dos
prédios, uma vez que estes dependem de visão direta (LOS -
\textit{Line-Of-Sight}) entre os satélites e o receptor. A reflexão do sinal
muitas vezes permite a leitura em ambientes fechados, porém o cálculo da posição
não será confiável \cite{Chen2000}. Portanto, apesar da ubiquidade dos
GNSSs em ambientes abertos, são necessárias soluções diferentes para obter um
Sistema de Posicionamento para Ambientes Fechados (IPS - \textit{Indoor
Positioning System}) sendo a ubiquidade deste essencial para conquistar o mesmo
nível de confiança trazido pelos GNSSs.

Para implementar este IPS, propomos o uso de tecnologias já implantadas em
dispositivos móveis e essenciais para o funcionamento dos mesmos, especialmente
as de camadas de comunicação, que são ubíquas no ambiente dos dispositivos
móveis, como Wi-Fi (padrão \textit{IEEE 802.11}) e Bluetooth (padrão \textit{Bluetooth
SIG}), para que os objetos conectados em que temos interesse de encontrar o
contexto locativo não necessitem de modificações.
