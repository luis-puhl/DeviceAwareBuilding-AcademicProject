\chapter{Aplicação demonstrativa}
\label{chap:aplicacao_demonstrativa}

Para a construção do software aplicativo, foi utilizado uma arquitetura em
três camadas: sensor, distribuidor de acesso (\emph{IoT gateway}) e apresentação
(Web). Nesta divisão, os sensores capturam as informações dos dispositivos
e repassam para a camada seguinte, no \emph{gateway} todas as partes se
encontram para fornecer e solicitar informações e, por último, a camada de
apresentação coleta o que é enviado dos sensores e gera uma página Web
para visualização dos dados capturados.

Esta divisão está de acordo com o padrão encontrado em outras aplicações
IoT onde a última camada usualmente varia entre apresentação e mineração
de dados (\emph{Data Mining}).

A camada de sensor utilizou as tecnologias Node.js, TShark parte
do Wireshark e MQTT.js. A camada \emph{gateway} foi composta
basicamente pelo \emph{MQTT Broker} Mosquitto. Por fim a camada de
apresentação utlizou as tecnologias Node.js, MQTT.js, HTML,
CSS, Javascript, Bootstrap e Google Maps API. A \autoref{fig-arq-app} apresenta
a arquitetura geral da aplicação demonstrativa.

\begin{figure}[htb]
	\caption{\label{fig-arq-app}Arquitetura geral da aplicação demonstrativa}
	\begin{center}
		\includegraphics[width=1\textwidth]{050-construcao/esquema-proj.png}
	\end{center}
	\legend{Fonte: Elaborada pelo autor}
\end{figure}
