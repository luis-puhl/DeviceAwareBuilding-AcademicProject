% ------------------------------------------------------------------------
% ------------------------------------------------------------------------
% Monografia Device Aware Building
% Trabalho de Conclusão de Curso
% Baseia-se no documento modelo de TCC do abntex2
% Para saber mais, acesse https://github.com/abntex/abntex2
% ------------------------------------------------------------------------
% ------------------------------------------------------------------------

\documentclass[
	% -- opções da classe memoir --
	12pt,				% tamanho da fonte
	openright,			% capítulos começam em pág ímpar (insere página vazia caso preciso)
	oneside,			% para impressão em verso e anverso. Oposto a oneside
	a4paper,			% tamanho do papel.
	% -- opções da classe abntex2 --
	chapter=TITLE,		% títulos de capítulos convertidos em letras maiúsculas
	%section=TITLE,		% títulos de seções convertidos em letras maiúsculas
	%subsection=TITLE,	% títulos de subseções convertidos em letras maiúsculas
	%subsubsection=TITLE,% títulos de subsubseções convertidos em letras maiúsculas
	% -- opções do pacote babel --
	english,			% idioma adicional para hifenização
	brazil				% o último idioma é o principal do documento
	]{abntex2}


% ----------------------------------------------------------
% Pacotes básicos
% ----------------------------------------------------------
%\usepackage{helvet}
\usepackage[scaled]{helvet}
\renewcommand*\familydefault{\sfdefault}	% Only if the base font of the document is to be sans serif (by Gabriel Oliveira)
\usepackage[T1]{fontenc}	% Selecao de codigos de fonte.
\usepackage[utf8]{inputenc}	% Codificacao do documento (conversão automática dos acentos)
\usepackage{lastpage}		% Usado pela Ficha catalográfica
\usepackage{indentfirst}	% Indenta o primeiro parágrafo de cada seção.
\usepackage{color}			% Controle das cores
\usepackage{graphicx}		% Inclusão de gráficos
\usepackage{microtype}		% para melhorias de justificação
% ----------------------------------------------------------

% ---
% Formatação de código-fonte
% ---
\usepackage{listings}
\usepackage{amsmath}


% ----------------------------------------------------------
% Pacotes adicionais, usados apenas no âmbito do Modelo Canônico do abnteX2
%% ----------------------------------------------------------
% \usepackage{lipsum}				% para geração de dummy text
\usepackage{000-sty/customizacoes}		% customizações feitas pelo autor
% ----------------------------------------------------------

% ----------------------------------------------------------
% Pacotes de citações
% ----------------------------------------------------------
\usepackage[brazilian,hyperpageref]{backref}	% Paginas com as citações na bibl
\usepackage[alf]{abntex2cite}					% Citações padrão ABNT

% ----------------------------------------------------------
% CONFIGURAÇÕES DE PACOTES
% ----------------------------------------------------------

% ----------------------------------------------------------
% Configurações do pacote backref
% ----------------------------------------------------------
\definecolor{thered}{rgb}{0.65,0.04,0.07}
\definecolor{thegreen}{rgb}{0.06,0.44,0.08}
\definecolor{thegrey}{gray}{0.5}
\definecolor{theshade}{rgb}{1,1,0.97}
\definecolor{theframe}{gray}{0.6}
% ----------------------------------------------------------

% Usado sem a opção hyperpageref de backref
\renewcommand{\backrefpagesname}{ }
% Texto padrão antes do número das páginas
\renewcommand{\backref}{\ABNTEXchapterfont}
% Define os textos da citação


% ----------------------------------------------------------
% Informações de dados para CAPA e FOLHA DE ROSTO
% ----------------------------------------------------------

\titulo{Habilitando um Prédio a Localizar Contextualmente Dispositivos utilizando Redes Sem Fio}
\autor{Luís Henrique Puhl de Souza}
\local{Bauru}
\data{2016}
\orientador{Prof. Dr. Eduardo Martins Morgado}

% O preambulo deve conter o tipo do trabalho, o objetivo,
% o nome da instituição e a área de concentração
% foi necessário utilizar \~{a} e etc para os acentos por problemas na geração do PDF

\instituicao{%
 Universidade Estadual Paulista ``Júlio de Mesquita Filho''
 \par
 Faculdade de Ciências - Campus Bauru
 \par
 Departamento de Computação
}

\tipotrabalho{Monografia (Trabalho de Conclusão de Curso)}

\preambulo{Trabalho de Conclus\~{a}o do Curso de Bacharelado em Ci\^{e}ncia da
Computa\c{c}\~{a}o apresentado ao Departamento de Computa\c{c}\~{a}o da
Faculdade de Ci\^{e}ncias da Universidade Estadual Paulista ``J\'{ú}lio de
Mesquita Filho'' – UNESP, C\^{a}mpus de Bauru.}

% ----------------------------------------------------------
% Configurações de projeto
% ----------------------------------------------------------
% MODIFICA A APRESENTACAO DAS PARTES OPCIONAIS
% ----------------------------------------------------------
% PARTE EXTERNA
%	Capa				(obrigatorio)
%	*Lombada			(opcional)
\newif\iflombada
\lombadafalse
% ELEMENTOS PRE-TEXTUAIS
%	Folha de rosto		(obrigatorio)
%	*Ficha Catalografica	(opcional) (obrigatoria para a UNESP-FC-BAURU)
\newif\ifficha
\fichatrue
%	*Errata			(opcional)
\newif\iferrata
\erratafalse
%	Folha Aprovacao		(obrigatorio)
%	*Dedicatoria		(opcional)
\newif\ifdedicatoria
\dedicatoriafalse
%	*Agradecimentos		(opcional)
\newif\ifagradecimentos
\agradecimentostrue
%	*Epigrafe			(opcional)
\newif\ifepigrafe
\epigrafefalse
%	Resumo				(obrigatorio)
\newif\ifresumo
\resumotrue
%	Resumo ingles		(obrigatorio)
%	*Lista Ilustracoes	(opcional)
\newif\iffiguras
\figurastrue
%	*Lista Tabelas		(opcional)
\newif\ifcodigos
\codigostrue
%	*Lista Tabelas		(opcional)
\newif\iftabelas
\tabelastrue
%	*Lista Abreviaturas	(opcional)
\newif\ifabreviaturas
\abreviaturastrue
%	*Lista Simbolos		(opcional)
\newif\ifsimbolos
\simbolosfalse
%	Sumario				(obrigatorio)
% ELEMENTOS TEXTUAIS
%	Introducao
%	Desenvolvimento		(capitulos e subcapitulos)
%	Conclusao
% ELEMENTOS POS-TEXTUAIS
%	Referencias			(obrigatorio)
%	*Glossario			(opcional)
\newif\ifglossario
\glossariofalse
%	*Apendice			(opcional)
\newif\ifapendice
\apendicetrue
%	*Anexo				(opcional)
\newif\ifanexo
\anexofalse
%	*Indice		(opcional)
\newif\ifindice
\indicefalse
% ----------------------------------------------------------

% ----------------------------------------------------------
% Configurações de aparência do PDF final
% ----------------------------------------------------------

% alterando o aspecto da cor azul
\definecolor{blue}{RGB}{0,0,0}

% informações do PDF
\makeatletter
\hypersetup{
		%pagebackref=true,
		pdftitle={\@title},
		pdfauthor={\@author},
		pdfsubject={\imprimirpreambulo},
		pdfcreator={LaTeX with abnTeX2},
		pdfkeywords={iot}{raspberry pi}{internet das coisas}{abntex2}{trabalho acadêmico},
		colorlinks=true,			% false: boxed links; true: colored links
		linkcolor=blue,			% color of internal links
		citecolor=blue,				% color of links to bibliography
		filecolor=magenta,			% color of file links
		urlcolor=blue,
		bookmarksdepth=4
}
\makeatother
% ----------------------------------------------------------

% ----------------------------------------------------------
% Espaçamentos entre linhas e parágrafos
% ----------------------------------------------------------

% O tamanho do parágrafo é dado por:
\setlength{\parindent}{1.3cm}

% Controle do espaçamento entre um parágrafo e outro:
\setlength{\parskip}{0.2cm} % tente também \onelineskip

% ----------------------------------------------------------
% compila o indice
% ----------------------------------------------------------
\makeindex
% ----------------------------------------------------------


% ----------------------------------------------------------


% ----------------------------------------------------------
% Início do documento
% ----------------------------------------------------------
\begin{document}

% Seleciona o idioma do documento (conforme pacotes do babel)
%\selectlanguage{english}
\selectlanguage{brazil}

% Retira espaço extra obsoleto entre as frases.
\frenchspacing

% ----------------------------------------------------------
% ELEMENTOS PRÉ-TEXTUAIS
% ----------------------------------------------------------
\pretextual


% ----------------------------------------------------------
% Capa
% ----------------------------------------------------------
\imprimircapa
% ----------------------------------------------------------


% ----------------------------------------------------------
% Folha de rosto
% (o * indica que haverá a ficha bibliográfica)
% ----------------------------------------------------------
\imprimirfolhaderosto
% ----------------------------------------------------------


% ----------------------------------------------------------
% Inserir a ficha catalográfica
% ----------------------------------------------------------

% ----------------------------------------------------------
% Inserir a ficha catalográfica
% ----------------------------------------------------------

% Isto é um exemplo de Ficha Catalográfica, ou "Dados internacionais de
% catalogação-na-publicação''. Você pode utilizar este modelo como referência.
% Porém, provavelmente a biblioteca da sua universidade lhe fornecerá um PDF
% com a ficha catalográfica definitiva após a defesa do trabalho. Quando estiver
% com o documento, salve-o como PDF no diretório do seu projeto e substitua todo
% o conteúdo de implementação deste arquivo pelo comando abaixo:
%
% \begin{fichacatalografica}
%	\includepdf{fig-ficha_catalografica.pdf}
% \end{fichacatalografica}

\ifficha
	\begin{fichacatalografica}
		\vspace*{\fill}					% Posição vertical
		\begin{center}					% Minipage Centralizado
			\fbox{
				\begin{minipage}[c][8cm]{13.5cm}		% Largura
					\hspace{0.5cm}
					\begin{minipage}{12.5cm}
						\small\ttfamily
						% \imprimirautor
						Puhl, Luís Henrique.
						%Sobrenome, Nome do autor

						\hspace{0.5cm} \imprimirtitulo \hspace*{1pt} / \imprimirautor, \imprimirdata

						\hspace{0.5cm} \pageref{LastPage} p. : il. \\

						\hspace{0.5cm} \imprimirorientadorRotulo~\imprimirorientador\\

						\hspace{0.5cm} \imprimirtipotrabalho ~--~
						Universidade Estadual Paulista. Faculdade de Ciências,
						\imprimirlocal, \imprimirdata\\

						\hspace{0.5cm} 1. Internet das Coisas.
						2. Raspberry Pi.
						3. Localização Contextual.
						4. MQTT.
						5. Node.js.
						6. TShark.
						7. Wi-Fi.
						I. Universidade Estadual Paulista "Júlio de Mesquita Filho". Faculdade de Ciências.
						II. Título
					\end{minipage}
					\hspace*{0.5cm}
				\end{minipage}
			}
		\end{center}
	\end{fichacatalografica}
	\vspace{2cm}
\fi
% ----------------------------------------------------------


% ----------------------------------------------------------
% Inserir errata
% ----------------------------------------------------------
%\begin{errata}
%Elemento opcional da \citeonline[4.2.1.2]{NBR14724:2011}. Exemplo:

%\vspace{\onelineskip}

%FERRIGNO, C. R. A. \textbf{Tratamento de neoplasias ósseas apendiculares com
%reimplantação de enxerto ósseo autólogo autoclavado associado ao plasma
%rico em plaquetas}: estudo crítico na cirurgia de preservação de membro em
%cães. 2011. 128 f. Tese (Livre-Docência) - Faculdade de Medicina Veterinária e
%Zootecnia, Universidade de São Paulo, São Paulo, 2011.

%\begin{table}[htb]
%\center
%\footnotesize
%\begin{tabular}{|p{1.4cm}|p{1cm}|p{3cm}|p{3cm}|}
% \hline
% \textbf{Folha} & \textbf{Linha} & \textbf{Onde se lê} & \textbf{Leia-se} \\
%	\hline
%	1 & 10 & auto-conclavo & autoconclavo\\
% \hline
%\end{tabular}
%\end{table}

%\end{errata}
% ----------------------------------------------------------


% ----------------------------------------------------------
% Inserir folha de aprovação
% ----------------------------------------------------------

% ----------------------------------------------------------
% Inserir folha de aprovação
% ----------------------------------------------------------

% Isto é um exemplo de Folha de aprovação, elemento obrigatório da NBR
% 14724/2011 (seção 4.2.1.3). Você pode utilizar este modelo até a aprovação
% do trabalho. Após isso, substitua todo o conteúdo deste arquivo por uma
% imagem da página assinada pela banca com o comando abaixo:
%
% \includepdf{folhadeaprovacao_final.pdf}
%
\begin{folhadeaprovacao}
	\begin{center}
		{\ImprimirAutor}

		\vspace*{\fill}\vspace*{\fill}
		\begin{center}
			\bfseries\large\ImprimirTitulo
		\end{center}
		\vspace*{\fill}

		\hspace{.45\textwidth}
		\begin{minipage}{.5\textwidth}
			\imprimirpreambulo
		\end{minipage}%
		\vspace*{\fill}
	\end{center}

	Aprovado em \detokenize{13/02/2017}.

	\vspace*{\fill}
	\begin{center}
		\uppercase{Banca examinadora}
	\end{center}

	\assinatura{\textbf{\imprimirorientador} \\ Orientador}
	\assinatura{\textbf{Profa. Dra. Simone das G. D. Prado}}
	\assinatura{\textbf{Prof. Convidado}}
	%\assinatura{\textbf{Professor} \\ Convidado 3}
	%\assinatura{\textbf{Professor} \\ Convidado 4}
	\vspace*{\fill}

\end{folhadeaprovacao}
% ----------------------------------------------------------


% ----------------------------------------------------------
% Dedicatória
% ----------------------------------------------------------
\ifdedicatoria
	\begin{dedicatoria}
		\vspace*{\fill}
		\centering
		\noindent
		\emph{ Dedicatória }
		\vspace*{\fill}
	\end{dedicatoria}
\fi
% ----------------------------------------------------------


% ----------------------------------------------------------
% Agradecimentos
% ----------------------------------------------------------
\ifagradecimentos
	\begin{agradecimentos}

		Agradeço a minha mãe, Bernardete Maria Puhl, e minha família pelo amor,
		apoio e incentivo que me acompanham desde sempre.

		Agradeço ao meu orientador Prof. Dr. Eduardo Morgado, por todo o tempo e
		trajetória no LTIA\footnote{Laboratório de Tecnologia da Informação
		Aplicada}, pela confiança, apoio e incentivo.

		A Carol Junqueira, pelo apoio, incentivo, paciência,
		companhia na correção e melhorias na construção desse documento.

		A todos os colegas do laboratório e de universidade, que sempre estiveram
		dispostos a orientar e aprender em grupo.

		Aos colegas V. Figueiredo, M. Cordeiro, K. Kimiko, H. Sumitomo, G. Oliveira,
		A. Peixinho, F. Avelar, E. Carreira, D.S. Santos, F. Beline, F. Lopes, M. Batista
		entre outros que ajudaram em minha formação técnica.

		Aos amigos D. Alexandre, F. Braga, R. Devellis, G. Galdino,
		entre outros que ajudaram em minha formação pessoal.

		A M. Barbosa pela opotunidade.

		A todos os professores, pois com muito esforço diário passaram não
		somente as informações técnicas e práticas de cada disciplina, mas
		lições que servirão para toda vida.

	\end{agradecimentos}
\fi
% ----------------------------------------------------------


% ----------------------------------------------------------
% Epígrafe
% ----------------------------------------------------------
\ifepigrafe
	\begin{epigrafe}
		\vspace*{\fill}
		\begin{flushright}
			Epigrafe
		\end{flushright}
	\end{epigrafe}
\fi
% ----------------------------------------------------------


% ----------------------------------------------------------
% RESUMOS
% ----------------------------------------------------------
\ifresumo
	% resumo em português
\setlength{\absparsep}{18pt} % ajusta o espaçamento dos parágrafos do resumo
\begin{resumo}

	IoT é o foco de empresas e entusiastas devido ao seu incrível crescimento com
	milhares de novos dispositivos todos os dias. Tudo isso construído sobre os
	baixos custos de processamento tanto em pequenos dispositivos quanto em
	grandes nuvens e da capacidade comunicacional que é cada vez mais exigida e
	presente em coisas do dia-a-dia.
	Através da exploração de plataformas emergentes (como o ESP8266 e o Raspberry
	Pi) e da construção de protótipos, este trabalho teve como objetivo
	construrir um sensor que permita que um prédio localize contextualmente
	qualquer dispositivo que se comunique utilizando Wi-Fi. Para
	alcançar esse objetivo, utilizou-se diversas ferramentas tencnológicas,
	incluindo Raspberry Pi 3, TShark, Node.js e MQTT. Estas ferramentas
	possibilitaram testes onde confirmou-se que não é possível associar uma
	distância geográfica à potência de sinal recebida (RSS) no caso de
	comunicações Wi-Fi, porém, com o mesmo sensor, é possível associar um
	dispositivo ao contexto de um sensor como uma sala dentro de um prédio.

	\textbf{Palavras-chave}: Internet das Coisas. Raspberry Pi. Localização Contextual. MQTT. Node.js. TShark. Wi-Fi.
\end{resumo}

	% resumo em inglês
\begin{resumo}[Abstract]
\begin{otherlanguage*}{english}

	IoT is at the focus of companies and enthusiasts due to its incredible
	growth with thousands of new devices every day. All built on top of the low
	processing costs (in both small hardware and large clouds) and the
	communicational capacity that is increasingly required by businesses and
	consumers alike and present in everyday things.
	Through the exploration of emerging platforms (such as ESP8266 and Raspberry
	Pi) and the construction of prototypes, this work aimed to construct a
	sensor that allows a building to contextually locate any device that
	communicates using Wi-Fi.
	To achieve this goal, several technological tools were used, including
	Raspberry Pi 3, TShark, Node.js and MQTT. These tools enabled tests where it
	was confirmed that it is not possible to associate a geographic distance to
	received signal strength (RSS) in the case of Wi-Fi communications, but
	with the same sensor we conclude that it is possible to associate a device
	with the context of that sensor such as at a room inside a building.

	\textbf{Keywords}: Internet of Things. Raspberry Pi. Contextual location. MQTT. Node.js. TShark. Wi-Fi
\end{otherlanguage*}
\end{resumo}

\fi
% ----------------------------------------------------------


% ----------------------------------------------------------
% inserir lista de ilustrações
% ----------------------------------------------------------
\iffiguras
	\pdfbookmark[0]{\listfigurename}{lof}
	\listoffigures*
	\cleardoublepage
\fi
% ----------------------------------------------------------


% ---
% inserir lista de listings
% ---
\ifcodigos
	\pdfbookmark[0]{\lstlistlistingname}{lol}
	\begin{KeepFromToc}
	\lstlistoflistings
	\end{KeepFromToc}
	\cleardoublepage
\fi
% ---

% ----------------------------------------------------------
% inserir lista de tabelas
% ----------------------------------------------------------
\iftabelas
	\pdfbookmark[0]{\listtablename}{lot}
	\listoftables*
	\cleardoublepage
\fi
% ----------------------------------------------------------


% ----------------------------------------------------------
% inserir lista de abreviaturas e siglas
% ----------------------------------------------------------
\ifabreviaturas
	\begin{siglas}
		\item[AP]		Ponto de Acesso Wi-Fi
		\item[AT]   	\emph{ATtention}
		\item[API]		\emph{Application Programming Interface}, conjunto de definições para comunicação entre componentes de software
		\item[CSV]		\emph{comma separated values} - valores separados por vírgula
		\item[dBm]		Unidade de medida para telecomunicações que expressa potência absoluta
		\item[DIY]		\emph{Do it by yourself} - Faça você mesmo
		\item[FSPL]		\emph{Free-space path loss} - perca no caminho em espaço aberto
		\item[GNSS]		  \emph{Global Navigation Satellite System} - Sistemas de navegação por satélite
		\item[IEEE]		Instituto de Engenheiros Eletricistas e Eletrônicos
		\item[IoT]		\emph{Internet of Things} - Internet das Coisas
		\item[IPS]		\emph{Indoor Positioning System} - Sistema de Posicionamento Interno
		\item[LBS]		\emph{Location-Based Services} - Serviços baseados em localização
		\item[LTIA]		Laboratório de Tecnologia da Informação Aplicada
		\item[MAC]		\emph{Media Access Control} - protocolo que coordena endereços de máquina a nível da camada de enlace de dados
		\item[MQTT]		\emph{Message Queue Telemetry Transport}
		\item[MU]		  \emph{Mobile User} - Usuário Móvel
		\item[NFC]		  \emph{Near Field Communication} - Comunicação Por Campo de Proximidade
		\item[PS]		  \emph{Positioning System} - Sistema de Posicionamento
		\item[RF]		  Radiofrequência
		\item[RFID]		  \emph{Radio-Frequency IDentification} - Identificação por radiofrequência
		\item[RP]		\emph{Reference Point} - Ponto de Referência
		\item[RPI3]		Raspberry Pi 3 model B
		\item[RSS]		\emph{Received Signal Strength} - Potência de Sinal Recebido
		\item[SSH]		\emph{Secure Shell} - Conexão segura entre terminais \emph{bash}
		\item[ToA]		  \emph{Protocol Time of Arrival}
		\item[Wi-Fi]	Marca registrada da Wi-Fi Alliance. Rede local sem fios baseados no padrão IEEE 802.11
	\end{siglas}
\fi
% ----------------------------------------------------------


% ----------------------------------------------------------
% inserir lista de símbolos
% ----------------------------------------------------------
\ifsimbolos
	\begin{simbolos}
			\item[$ \Lambda $] Lambda
	\end{simbolos}
\fi
% ----------------------------------------------------------


% ----------------------------------------------------------
% inserir o sumario
% ----------------------------------------------------------
\pdfbookmark[0]{\contentsname}{toc}
\tableofcontents*
\cleardoublepage
% ----------------------------------------------------------


% -----------------------------------------------------------------------------
% ELEMENTOS TEXTUAIS
% -----------------------------------------------------------------------------
\textual


\chapter[Introdução]{Introdução}
%\addcontentsline{toc}{chapter}{Introdução}

Nos recentes anos de 2014 a 2016, a Internet das Coisas (IoT - \emph{Internet of
Things}) vem tomando o foco das atenções de empresas e entusiastas de Tecnologia
da Informação \cite{DzoneIoT:2015} e, como é esperado que uma quantia total de
6,4 bilhões de dispositivos conectados exista até o final de 2016
\cite{GARTNER2015} e entre 26 bilhões \cite{GARTNER2014} e 50 bilhões até 2020
com até 250 novas coisas conectando-se por segundo \cite{CiscoBlog2013}, as
empresas líderes do segmento já incluem IoT como uma de suas áreas de atuação
\cite{Ibm2016, ARM-mbed, Microsoft2016, Intel2016, Oracle2016, Google2016,
AmazonIoT2016}.

Todo este movimento no mercado é justificado pelo baixo custo dos pequenos
dispositivos computacionais \cite{RpiZeroLaunch, Esp8266.net} e grandes serviços
na nuvem \cite{Kaufmann2015, Amazon2016}. Este baixo custo possibilita a
computação ubíqua descrita por \citeonline{Weiser1991c} que nesta obra é
entendida como \emph{``computação onipresente diluída no dia-a-dia''}. Também
nesta obra, esta onipresença diluída no plano de fundo é a  base e a
consequência para o conceito e área de IoT, sendo esta a realizadora da
computação ubíqua.

Uma vez contextualizado o mercado e a oportunidade de implementação da
computação ubíqua, percebe-se a necessidade de dar aos elementos cotidianos
(coisas) a capacidade info-computacional, tornando-os sensores e atuadores
conectados, unicamente identificáveis e acessíveis através da rede mundial de
computadores \cite{Lemos2013, Kranenburg2012}. Para tanto, este trabalho propõe
a construção de um sensor que, através da rede, identifica e localiza contextualmente
os elementos cotidianos.

\section{Problema}
\label{sec:Problema}

Tamanha quantidade de dispositivos conectados pouco acrescenta na vida diária se
humanos ou coisas não puderem simplesmente se encontrar, tanto em ambiente real
quanto virtual é necessário o contato entre as partes para a existência de uma
interação.

Mais ainda, para melhor funcionamento de aplicações como o uso de conteúdo
específico feito sob medida para cada usuário e situação é necessário coletar
informações contextuais. Para a maioria das aplicações, a informação contextual
de maior relevância é a localização física.

Este tipo de situação destaca a necessidade da criação desta informação através
de sensores ativos sempre que necessário para que o dispositivo tenha ciência
deste contexto em suas tomadas de decisão e para que outros (sistemas, pessoas e
coisas) saibam a localização de qualquer dispositivo ao qual têm interesse de
interagir.

Um exemplo da necessidade de localização de dispositivos dentro de um prédio
seria um profissional saber onde está o dispositivo em seu local de trabalho,
seja ele um vendedor e seu tablet para demostrar um produto fora de estoque em
uma loja ou um médico e um desfibrilador.

\subsection{Sobre Sistemas de Posicionamento}
\label{subsec:Sobre Sistemas de Posicionamento}

Sistemas de posicionamento (PS - \emph{Positioning System}) são geralmente
constituídos de um Ponto Origem Global escolhido (\emph{O}) e um conjuto não
vazio de Pontos de Referência (RP - \emph{Reference Point}) cuja localização
global em relação ao \emph{O} é conhecida com precisão maior ou igual a
oferecida pelo sistema.

Também faz parte do sistema o ponto móvel (MU - \emph{Mobile User}) sobre o
qual temos interesse em determinar a posição que é feita pelo PS encontrando uma
distância (com dimensão variável de acordo com o método utilizado para adquirir
a distância) relativa a um sub-conjunto de RPs. Feito isso, é possível utilizar
modelos matemáticos para, a partir das distâncias, encontrar uma posição do MU
em relação aos RPs e uma nova transformação é aplicada para encontrar a posição
relativa ao \emph{O}.

Uma das maneiras de classificar PSs é entre os de Auto Posicionamento e
Posicionamento Remoto. Os de Auto Posicionamento contém no MU todo aparato
necessário para medir a distância dos RPs e calcular a posição em relação a
\emph{O}. Já os de Posicionamento Remoto tem o mínimo necessário na MU e todo
o trabalho de cálculo de distância e posição global é feito nos RPs ou em uma
unidade coordenadora destes.

Para PSs eletrônicos baseados em radio-frequência (RF - \emph{Radio
Frequency}), geralmente, utilizam-se dois componentes básicos, Transmissores e
Receptores, os quais assume-se que ao menos um destes está no RP e ao menos um
outro no MU. Para calcular a distância entre MU e RP, utiliza-se as propriedades
da comunicação por RF como tempo de chegada (TOA - \emph{Time Of Arrival}),
diferencial de tempo de chegada (TDOA - \emph{Time Difference Of Arrival}) e
ângulo de chegada de sinal (AOA - \emph{Angle Of Arrival}).

Para maior precisão, é comum a utilização de múltiplas RPs geralmente com o
número mínimo igual ao número de dimensões espaciais que deseja-se calcular.
Nota que para sistemas distribuídos a sincronização de relógios é um problema
intrínseco então é fundamental que o tempo seja contado como dimensão.

Os sistemas classificados como ``Sistema de Navegação Global por Satélite''
(GNSS - \emph{Global Navigation Satellite System}), como o tradicional
Estadunidense Sistema de Posicionamento Global (GPS - \emph{Global Positioning
System}), utilizam a técnica em que o dispositivo móvel contém o receptor e os
transmissores são fixos em satélites na órbita terrestre \cite{Djuknic2001}.
Devido a posição e número de satélites, o GPS e seus correlatos estão sempre
presentes do ponto de vista de um observador da superfície terrestre, sendo para
este tipo de usuário um sistema ubíquo.

Entretanto, a força do sinal GNSS não é suficiente para penetrar a maioria dos
prédios, uma vez que estes dependem de visão direta (LOS -
\emph{Line-Of-Sight}) entre os satélites e o receptor. A reflexão do sinal
muitas vezes permite a leitura em ambientes fechados, porém o cálculo da posição
não será confiável \cite{Chen2000}. Portanto, apesar da ubiquidade dos
GNSSs em ambientes abertos, são necessárias soluções diferentes para obter um
Sistema de Posicionamento para Ambientes Fechados (IPS - \emph{Indoor
Positioning System}) sendo a ubiquidade deste essencial para conquistar o mesmo
nível de confiança trazido pelos GNSSs.

Para implementar este IPS, propoem-se o uso de tecnologias já implantadas em
dispositivos móveis e essenciais para o funcionamento dos mesmos, especialmente
as de camadas de comunicação, que são ubíquas no ambiente dos dispositivos
móveis, como \emph{Wi-Fi} (padrão \emph{IEEE 802.11}) e \emph{Bluetooth}
(padrão \emph{Bluetooth SIG}), para que os objetos conectados no qual tem-se
interesse de encontrar o contexto locativo não necessitem de modificações.

Outros protocolos de comunicação sem fio ubiquos existem (em especial o
celulares em todas as gerações 2G, 3G, 4G) porém não oferecem a mesma
flexibilidade por trabalharem em uma faixa de radio-frequência licenciada e por
questões de propriedade da rede que serão abordadas na seção de Localização
Contextual desta mesma obra.

De forma semelhante, existem protocolos mais flexíveis (nas faixas não
licenciadas como \emph{NFC}, infra-vermelho, \emph{ZigBee} ou
\emph{SIGFOX}) porém estes não estão presentes na maiora dos aparelhos
utilizados tanto globalmente quanto localmente removendo a característica da
forma de comunicação ubíquoa que é foco deste trabalho.

Devido as restrições anteriores justifica-se o foco nas tecnologias de
comunicação \emph{Wi-Fi} e \emph{Bluetooth} porém trabalhar com as duas
tecnologias simultaneamente é um problema complexo por si só, então, a escolha
de um ou outro apesar de a nível global serem de equivalente importância para
esta obra (ambas tem mesma importância e presença no mercado atual, permitem
flexibilidade por possuirem protocolos conhecidos por todos em frequencias
livres de licenciamento, dentro da área de cobertura que são de nosso interesse
e o usuário final já ser o proprietário da rede local criada) deve ser feita.
Esta escolha toma um único parametro como decisivo que é a observação do
ambiente de teste do protótipo onde pouco existe o uso de \emph{Bluetooth} que
reflete o costume local de mante-lo desligado e em comparação com \emph{Wi-Fi}
que está sempre ligado em todos dispositivos, conectando os mesmos diretamente à
internet. Portanto \emph{Wi-Fi} é a tecnologia de maior interesse por pequena
margem.

\section{Motivação}
\label{sec:Motivação}

A proposta deste trabalho é criar um ambiente contextual, onde a localização
contextual oriunda do posicionamento remoto de cada dispositivo móvel é
administrada e divulgada pelo prédio conectado ao invés da auto-localização do
aparelho, pois:

\begin{alineas}

	\item Uma vez encontrada a localização, é mais fácil propagar esta informação do
ambiente para o aparelho em comparação ao autoposicionamento, pois a negociação
entre o ambiente e o aparelho é nula quando o primeiro contém a informação- o
ambiente sempre disponibilizará uma informação coletada para o gerador desta
informação;

	\item Pode-se lidar com grande heterogeneidade de dispositivos, uma vez
que cada um deles não precisa se adaptar para cada mudança de ambiente;

	\item Este tipo de informação já é contida nos históricos de cada Ponto de
	Acesso Wi-Fi (AP), porém:

	\begin{alineas}

		\item Geralmente sem uso - poucas são as aplicações que usam a
		localização obtida pelo AP;

		\item Com granularidade insuficiente para uso em aplicações
		contextualizadas;

		\item geralmente não disponibilizada pelos APs.

	\end{alineas}

	\item Uma vez instalado um PS deste gênero, a quantia de dispositivos que
	ele pode localizar fica limitada apenas pela rede física anteriormente
	instalada;

	\item Economia de hardware quando menos é exigido de cada dispositivo móvel.

\end{alineas}

Nota-se também que mesmo com a quantidade prevista de 5 dispositivos IoT por
pessoa em média, estes seriam beneficiados sempre que utilizados no ambiente
conectado proposto.

\begin{figure}[htb]
	\caption{\label{fig-projeto}Modelo das camadas }
	\begin{center}
		\includegraphics[width=1\textwidth]{012-justificativa/img/projeto.jpg}
	\end{center}
	\legend{Fonte: \citeonline{cordeiro2016}}
\end{figure}

A Figura \ref{fig-projeto} apresenta a arquitetura simplificada de uma aplicação
IoT, e no detalhe inferior a relação deste projeto com o do aluno Marcelo Augusto
Cordeiro, também do Bacharelado de Ciências da Computação, que é também membro do
ambiente de testes LTIA (Laboratório de Tecnologia da Informação Aplicada) da Unesp de Bauru
e do mesmo edital para obter o título de bacharel.


\section{Objetivos}
\label{sec:Objetivos}

\subsection{Objetivo Geral}
\label{subsec:Objetivo Geral}

Considerando características locais, propõem-se a construção de uma aplicação
para localizar contextualmente dispositivos dentro de um prédio piloto e avaliar
sua precisão.

Além da aplicação, é objetivo definir o custo do projeto piloto, incluindo
esforço de pesquisa assim como definir um custo para replicação deste
localizador contextual em outros prédios utilizando como fonte de ferramentas e
recursos o mercado local.

\subsection{Objetivos Específicos}
\label{subsec:Objetivos Específicos}

\begin{alineas}

	\item Estabelecer o estado da arte sobre a desenvolvimento de aplicações IoT;

	\item Identificar desafios locais para o desenvolvimento;

	\item Identificar provedores de serviços, dispositivos e ferramentas para o
desenvolvimento;

	\item Construir sensores de identificação e localização (distância) de
 dispositivos cuja comunicação seja baseada em Wi-Fi;

	\item Posicionar estes sensores;

	\item Construir um dispositivo agregador de informações dos sensores
 (\emph{gateway}) e sua interface web (MQTT - \emph{MQ Telemetry Transport});

	\item Estimar o custo total do projeto piloto incluindo esforço de pesquisa;

	\item Estimar o custo de replicação da aplicação em outros prédios
	utilizando fontes do mercado local.

\end{alineas}


\chapter{Fundamentação teórica}
\label{chap:Fundamentação teorica}


Para conceituar, fundamentar e dar suporte teórico ao presente trabalho
apresentam-se neste capítulo os tópicos e definições dos segmentos: IoT,
localização contextual de dispositivos e localização baseada em redes sem fio.

\section{Internet das coisas (IoT)}
\label{sec:INTERNET DAS COISAS (IOT)}

Uma das primeiras aplicações e definições de IoT foi feita simultaneamente por Kevin Ashton em
1999 para a \emph{Procter \& Gamble} (P\&G) \cite{ASHTON2009} e
pelo laboratório Auto-ID Labs no Instituto de Tecnologia de
Massachusetts (MIT - \emph{Massachusetts Institute of Technology}) utilizando
identificação por radio-frequência (RFID - \emph{radio-frequency
identification}) \cite{ATZORI2010, Friedemann2011}. Desde então, a IoT cresceu
ultrapassando o escopo da tecnologia RFID, porém sempre com as premissas de ``uma
infraestrutura global para a Sociedade da Informação, habilitando serviços
avançados através da interconexão de coisas (físicas e virtuais) baseadas em
tecnologias, existentes e evolutivas, de informação e comunicação'' descrita por
\apudonline[p.~1, grifo e tradução
nossa]{Wortmann2015}{InternationalTelecommunicationUnion2012}.

Hoje em dia, quase qualquer tecnologia de comunicação acessível a computadores
pode ser utilizada como meio de comunicação entre dispositivos IoT, tornando
o RFID mais uma, porém de grande importância, tecnologia info-comunicacional a
disposição das coisas para sua conexão. Esta gama de tecnologias possibilita uma
variedade equivalente de coisas conectadas. Se a coisa pode usar de uma
tecnologia de conexão, considerando suas restrições de volume, custo e
utilidade, muito provavelmente vai fazê-lo gerando ao menos uma identidade
virtual representando seu objeto físico e seus atributos. Esta identidade
virtual e atributos virtuais serão expostos para todos indivíduos, humanos ou
coisas, que lhe forem convenientes de qualquer lugar do universo virtual,
fazendo efetivamente parte da Internet.

\section{Localização contextual de dispositivos}
\label{sec:Localização contextual de dispositivos}

Em ciência da computação, os termos \emph{"Contexto"} e \emph{"Consciência
de Contexto"} expressam uma ideia recente estudada nos campos de inteligência
artificial e ciência cognitiva desde 1991. O tema "Contexto" ainda é considerado
atual e promissor a ponto de mudar o cenário de negócios nos próximos 10 anos, mas
sem definição simples. Tamanha é a falta de uma definição geral que
realmente funcione para casos reais que existe uma proposta de definir o termo
utilizando uma nova metodologia de pesquisa holística através de mineração e
agrupamento de texto advindo de publicações científicas \cite{Pascalau2013}.

Mesmo sem uma definição permanente em vista, utilizou-se o que é considerado
estado da arte para o termo \emph{"Contexto"} que foi introduzido por
\citeonline{Dey1999} e reforçado por \citeonline{Dey2000}:

\begin{citacao}

	``Contexto é qualquer informação que pode ser utilizada para caracterizar a
	situação de uma entidade. Uma entidade é uma pessoa, lugar ou objeto que é
	considerado relevante para a interação entre um usuário e uma aplicação,
	incluindo o próprio usuário e a aplicação.'' \

	\citeonline[p.~3]{Dey1999} Tradução Nossa.
\end{citacao}

\subsection{Localização contextual}
\label{subsec:Localização contextual}

Das informações contextuais que uma aplicação de cliente móvel pode obter, a
localização é uma das mais importantes. Ajudar pessoas a navegar por mapas,
encontrar objetos e pessoas com os quais tem interesse de interagir é sem dúvida
uma boa meta a ser alcançada com a coleta da localização do cliente
\cite{Bellavista2008}.

Na categoria de Serviços Baseados em Localização (LBS - \emph{Location-Based
Services}) existem duas gerações. A primeira orientada a conteúdo que falhou,
pois a informação de localização era armazenada pela rede (que geralmetne era
administrada por uma empresa de telecomunicações), podendo até ser vendida pelo
provedor a terceiros, causando a sensação de \emph{Spam} (conteúdo não
solicitado) no usuário final ao receber conteúdo desta provedora. Já na segunda
geração, a posse da informação foi movida para o cliente móvel, deixando a cargo
do usuário escolher se ela seria compartilhada e com quem. Esta mudança trouxe
maior engajamento do usuário, resultando numa maior aceitação dessa geração
\cite{Bellavista2008}.


Ao contrário das técnicas atuais, neste trabalho os humanos ou tomadores de
decisão não estarão em posse do cliente móvel, e sim em posse do prédio.
Portanto, a mesma informação, sem degradação em sua importância, passará a ser
coletada e armazenada pelo provedor da rede como nos LBSs de primeira geração.
Esta decisão garante o foco no usuário uma vez que este mudou, antes ele detinha
um cliente móvel, agora ele detem múltiplos. Isso torna a detenção do todo
(coisas dentro do prédio) mais precioso do que o das partes (os clientes móveis)
além da mudança da propriedade da rede para o usuário final, na comparação
celular \emph{versus} \emph{Wi-Fi}.

Uma vez encontrada a localização de um dispositivo, metadados sobre o prédio são
mesclados formando um conjunto rico contextualmente do ponto de vista da
aplicação IoT Prédio como fornecedora principal dos dados para a Internet e,
portanto, seus usuários detentores. Essa riqueza é garantida com metadados sobre
o dispositivo (identificação, nome, histórico, carecterísticas) e sobre o prédio
(ex.: mapa, estrutura de salas, humanos responsáveis e lista de equipamentos) que trazem possibilidades de
extração de informação importantes para os detentores deste prédio e seu
conteúdo. Esta capacidade do prédio deve-se pelo papel de coordenador de
informações e controlador de meta-informações semelhante ao Coordenador em uma
aplicação na arquitetura Modelo-Apresentação-Adaptador-Controlador-Coordenador
(MPACC - \emph{Model-PresentationAdapter-Controller-Coordinator}) proposto por
\citeonline{Roman2001}.


\subsection{Contexto de um dispositivo em um prédio}
\label{subsec:Contexto de um dispositivo em um prédio}

Para metadados agregados à informação de posição pelo prédio defini-se que, para uma aplicação IoT, o
modelo de divulgação tem de conter além da posição do dispositivo informação
sobre este (nome, histórico), informação da estrutura do prédio, ligação entre a estrutura
do prédio e a localização do dispositivo e informação
sobre o estado do prédio.


Este modelo visa prover fácil mineração e reutilização de informações por
terceiros que é medida pela disponibilidade e
relacionamento das informações providas. Essa métrica também será utilizada para
avaliar o projeto.

Este foco em reusabilidade vem da definição de Web Semântica (\emph{Semantic
Web}) e de uma de suas realizadoras, a Ligação de Dados (\emph{Linked Data}),
que sugerem o uso de um formato padrão além de ser acessível e gerenciável pelas
ferramentas de exploração. Desta forma a Web de Dados (\emph{Web of Data}) é
construída opondo uma simples coleção de dados \cite{Bizer2009}.

\section{Localização baseada em redes sem fio}
\label{sec:Localização baseada em redes sem fio}

Um sistema de posicionamento pode ser baseado em técnicas
\emph{n-lateração?} de distâncias adquiridas com a medição de características
eletromagnéticas (ex.: potência de sinal) e dos protocolos (ex.: Tempo de
chegada) que já foram explorados anteriormente \cite{Abusubaih2007,
bahillo2009ieee, Feldmann2003}.

Portanto, os sensores seguem as especificações de \emph{WiFi IEEE 802.11}
\cite{Crow1997} e técnicas definidas para \emph{Bluetooth Low Energy (BLE)}
\cite{Hossain2007} devido a semelhança da área de cobertura (até 100 metros,
geralmente utilizado até 20 metros) e frequência (no caso de 2.4GHz).

Para construir estes sensores uma plataforma de hardware adequada é necessária,
para esta escolheu-se o Raspberry Pi \cite{Vujovic2014, Vujovic2015} que já
foi provado funcional no caso de Localização através \emph{Wi-Fi} por
\citeonline{Ferreira2016} especialmente a sua versão 3 que adiciona a capacidade
de sensor \emph{Wi-Fi} e \emph{Bluetooth} em sua placa principal sem
necessidade de adaptadores externos destacando ainda mais sua escolha
\cite{RPI2016}. Em adição, na construção dos sensores foi testada a plataforma ESP8266 bem como
outras alternativas que demonstraram afinidade com essas características.



\chapter{Método de Pesquisa}
\label{chap:Método de Pesquisa}

Abordagens para medir distâncias através de redes sem fio \emph{Wi-Fi}
\cite{bahillo2009ieee} e \emph{Bluetooth} já existem e propor novas maneiras
não é o foco deste trabalho. Utilizando essas técnicas, constitui-se uma
rede de nós sensores colaborativos fixos no ambiente onde deseja-se obter a
localização dos dispositivos. As informações de distância são compartilhadas
entre os nós para maior precisão da informação.

Para a implementação, utilizou-se os \emph{softwares} de maior
destaque recentemente nos ramos de comunicação de baixa energia (\emph{MQTT}),
serviços \emph{Web} para geolocalização (\emph{Google Maps}) e publicação
(\emph{NodeJS}), além de \emph{softwares} para medição da distância sem
interferir na comuncação (\emph{Sniffing}) e das plataformas de
\emph{hardware} disponíveis e recomendadas para IoT com capacidade
\emph{Wi-Fi} (\emph{Raspberry Pi 3} e \emph{ESP-8266}).

Mesmo com a grande quantidade de dispositivos já conectados são poucos os
documentos descrevendo boas práticas para concepção, construção e manutenção de
aplicações IoT, especialmente sobre os cuidados tomados quanto a segurança e
análise de custos para a implementação e manutenção. Além
disso, a falta de referências neste sentido é agravada quando considera-se a
implementação no interior do estado de São Paulo. Nesta região, poucas são as
organizações atualizadas neste tema, levando a uma falta enorme de conteúdo
escrito na linguagem local além de serviços e produtos disponíveis para
construção de uma plataforma completa e competitiva na região.

Devido a falta de conteúdo e instrução, utiliza-se prototipagem ágil neste
projeto, uma vez que esta metodologia de desenvolvimento é recomendada para
projetos cujas especificações e definições não são claras, demandando muitas
modificações das mesmas durante a etapa de execução. Esse método entra em
contraste com metodologias clássicas, como a cascata, que apesar de previsíveis,
não reagem bem a ambientes de extrema incerteza.

Mais especificamente, utiliza-se uma variante da metodologia \emph{Scrum}
\cite{James2016} que foi adaptada para o projeto. Nela, foram executadas
iterações de uma semana em que a cada iteração, uma nova versão melhorada do
produto completo (\emph{hardware}, \emph{software}, documentação e
resultados) foi feita.

Dentro de cada iteração, as camadas da aplicação IoT foram escolhidas,
implementadas, justificadas e avaliadas, sendo parte do processo registrado sob forma de
vídeo (Youtube).

A cada iteração, cumpriu-se parte ou todo de cada objetivo proposto no trabalho,
levando o projeto gradualmente para um estágio de completude.
Cada iteração teve como foco os objetivos a seguir, sendo seus resultados
utilizados para tomar e justificar decisões durante a execução do
projeto bem como servir de posterior documentação. Os objetivos de cada iteração
são:

\begin{alineas}

	\item Escolha de provedores de serviços, dispositivos e ferramentas para o
desenvolvimento;

	\item Construir, avaliar, testar e manter os sensores;

	\item Construir o dispositivo agregador e sua API;

	\item Estimar o custo total do projeto piloto;

	\item Estimar o custo de replicação;

	\item Identificar os desafios para o desenvolvimento.

\end{alineas}

Desta forma, a liberdade necessária foi garantida para o projeto ser
executado com sucesso, mesmo no ambiente de incerteza no qual o mercado local de
IoT encontra-se, cumprindo as premissas de funcionamento, manutenção e
segurança que são grande importância para os interessados na área.


\chapter{Plataformas}
\label{chap:Plataformas}

Para a localização com os resíduos de comunicação \emph{WiFi} são necessários
sensores que possam capturar estes resíduos e processar qualquer informação
capturada pelo sensor deste trabalho. Esta plataforma de sensor pode ser construída com
qualquer plataforma computacional capaz de ser programada com comunicação
\emph{WiFi}, porém o \emph{hardware} de \emph{WiFi} e seu \emph{software}
controlador deve permitir o Modo Promíscuo.

Este Modo Promíscuo (\emph{promiscuous mode}) é definindo pela capacidade de uma
Placa Adaptadora de Rede \emph{WiFi} (\emph{Network Interface Card} -
\emph{NIC}) receber e interpretar todos os pacotes que trafegam em uma rede ou
em todas as redes que estão em seu alcance, independentemente do destinatário do
pacote. Em seu fucionamento normal, uma \emph{NIC} descarta todos os pacotes que
não são destinados para ela o mais cedo possível, evitando reprocessamento de
dados indesejáveis, por este motivo não são todas as \emph{NICs} que permitem o
Modo Promíscuo. Essa funcionalidade elimina a necessidade de \emph{hardware} ou
\emph{software} em cada um dos dispositivos rastreados.

Neste sentido, elegeu-se duas plataformas de notável importância no mercado atual
e notável facilidade de acesso para qualquer interessado na área. As plataformas
testadas foram o microcomputador \emph{Raspberry Pi} e o microcontrolador
\emph{ESP8266}. Ambos  foram escolhidos pelo domínio do segmento de Prototipação
e Faça Você Mesmo  (\emph{Do It Yourself} - \emph{DIY}) dentro do campo de IoT.
Outro líder de segmento, o \emph{Arduino}  foi prontamente descartado por não
conter nativamente a habilidade de conectar-se à \emph{Internet} sendo
constantemte combinado com um dos escolhidos para ganhar esta habilidade,
demonstrando claramente menor afinidade a este projeto em comparação aos seus
igualmente famosos concorrentes. A seguir, serão apresentadas  essas duas plataformas
e como foi a experimentação com cada uma.

\section{ESP8266}
\label{sec:ESP8266}

O ESP8266 é um SOC (\emph{System On a Chip} - Sistema em um chip),
ou seja, é um chip com todos os componentes lógicos
eletrônicos necessários e partes para um dado sistema em único cirtuito
integrado. Este chip possui:


\begin{alineas}
	\item Wi-Fi embutido de 2,4 GHz (802.11 b/g/n);

	\item 16 GPIOs (\emph{general-purpose input/output}) incluindo interfaces
 I2U, SPI, UART, entrada ADC, saída PWM;

	\item Arquitetura RISC de 32 bits;

	\item CPU que opera em  80 MHz, com possibilidade de operar em 160 MHz;

	\item 64 KB de ROM para \emph{boot};

	\item 64 KB de RAM para instruções;

	\item 96 KB de RAM para dados;

	\item Memória Flash SPI de 512 KB a 4 MB (dependente de módulo externo);

	\item Núcleo baseado no IP Diamand Standard LX3 da Tensilica.

\end{alineas}

Para o mercado de prototipação, fabricantes constroem placas de diferentes configurações com
este chip como elemento central, os chamados módulos. Estes módulos usam o
ESP8266 com diferenças perceptíveis, por exemplo, quantidade de pinos, dimensões
físicas, alguns podem até operar de modo \emph{standalone} (sem outro hardware de
suporte como reguladores de tensão e conversores serial-USB) e, especialmente, a
 Memória Flash SPI. Neste trabalho, foram usados os módulos:
ESP-01, LoLin, D1 mini e ESP-12f com placa adaptadora de pinos.
Cada um deles pode ser encontrado na \autoref{fig:modelos-esp}.


\begin{figure}[htb]
	\caption{\label{fig:modelos-esp}Módulos ESP8266}
	\begin{center}
		\includegraphics[width=1\textwidth]{040-plataformas/esp-dev/modulos-esp.jpg}
	\end{center}
	\legend{Fonte: Elaborada pelo autor}
\end{figure}


\clearpage
\subsection{Disponibilidade no mercado}
\label{subsec:mercado-esp}

As diferentes especificações implicam em diferentes produtos e mercado para
eles, isto resulta em diferentes custos em diferentes regiões.

\begin{table}[htb]
\IBGEtab{%
\ABNTEXchapterfont {
  \caption{Descrição e custos de módulos ESP8266}%
  \label{table:custo-esp}
}
}{%
\begin{tabular}{cccc}
\toprule
Módulo				&	Pinos de GPIO e conectores							&	Memória	&	Custo			\\
\midrule \midrule
ESP-01				&	8 pinos macho, incompatível com \emph{breadboard}	&			&  \\
					&	(GND, 3v3, TX, RX, CH_PD, RST, GPIO 0, GPIO 2)		&	1 MB	&	R\$ 16,80 		\\
\midrule
ESP-12f				&	22 pontos para montagem em superficie, nenhum pino	&	4 MB	&	R\$ 14,90 		\\
\midrule
D1 mini (ESP-12f)	&	16 + microUSB										&	4 MB	&	R\$ 12,56^{1}	 \\
\midrule
LoLin (ESP-12f)		&	30 + microUSB										&	4 MB	&	R\$ 35,87 		\\
\midrule
\bottomrule
\end{tabular}%
}{%
	\fonte{Produzido pelo autor.}%
	\nota[Nota 1]{D1 mini (ESP-12f) foi adquirido do mercado chinês.}%
}
\end{table}

\begin{table}[htb]
\IBGEtab{%
\ABNTEXchapterfont {
  \caption{Descrição e custos de acessórios para ESP8266}%
  \label{table:acessorios-esp}
}
}{%
\begin{tabular}{ccc}
	\toprule
	Acessórios						&	Descrição									&	Custo			\\
	\midrule
	Esp8266 Placa Para Soldar		& Placa com 16 pinos conectados aos				&					\\
	Esp-07, Esp-08, Esp-12, Esp-12e	& pontos de superficie do ESP-12f				&	R\$ 3,45 		\\
	\midrule
	Conversor Usb Serial 			&	Fornece uma conexão serial-USB entre		& 	 				\\
	Ch340 Rs232 - 3,3v 5v^{1}		&	o ESP8266 e o computador de desenvolvimento	&	R\$ 6,87 		\\
	\midrule
	Adaptador Usb Serial			&	Fornece uma conexão serial-USB entre		&		\\
	Ttl Conversor Cp2102^{2}		&	o ESP8266 e o computador de desenvolvimento	&	R\$ 20,00		\\
	\midrule
	Ams1117 3,3v (3.3v) - Lm1117	&	Regula a tensão de uma USB ou pilhas para	&		\\
									&	3.3\emph{V} 1\emph{A} usado nos módulos		&	R\$ 1,50 \\
	\midrule
	Fonte Usb 5v 2a Celular			&	Fonte de alimentação com padrão USB			&		\\
	Gps Android Ipod				&	de 5\emph{V} utilizada com D1 mini	 		&	R\$ 9,90 \\
	\midrule
\bottomrule
\end{tabular}%
}{%
	\fonte{Produzido pelo autor.}%
	\nota[Nota 1]{Compatível apenas com \emph{Windows 7}.}%
	\nota[Nota 2]{Compatível com \emph{Windows 10} e com o computador de desenvolvimento.}%
}
\end{table}


O ESP8266 foi escolhido como primeira tentativa devido ao seu baixo custo e ao
tamanho reduzido. No exterior, ele pode ser encontrado de USD\$ 1,76 a 2,2
\cite{AlibabaESP}, e no Brasil por aproximadamente BRL R\$ 15,00 \cite{mercadolivreEsp}.

Devido ao seu tamanho, ele é de fácil integração com demais dispositivos,
bastando o uso de uma comunicação serial. Já sobre a comunidade, há inúmeros
projetos DIY que ensinam a como construir e manipular projetos que envolvem diferentes
módulos. Além disso, a empresa  idealizadora e fabricante do chip, Espressif,
disponibiliza no GitHub projetos com documentação e código aberto.

Para desenvolver na plataforma, os módulos ESP8266 foram utilizados de formas
diferentes dependendo das capacidades de um deles. Quando o módulo possuía
regulador de tensão embarcado, utilizava-se o próprio conectado a uma porta USB.
Quando o módulo não possuía tal, utilizava-se um circuito com fonte externa
(pilhas ou USB) e um regulador de tensão conectados aos pinos 3v3 e
GND. Depedendo da complexidade do circuito para ligar e ter acesso à
serial do módulo, foi necessário o uso de uma placa \emph{breadboard}, como na
\autoref{fig:esp-pilha-serial}. Para este trabalho foi utilizado o regular
AMS1117 3v3 e dois capacitores de $100 \mu F$. Nas
\autoref{table:custo-esp} e \autoref{table:acessorios-esp} são apresentados os
custos de cada módulo testado e dos acessórios utilizados em conjunto.


\subsection{Desenvolvimento e Implantação}
\label{subsec:dev-esp}

Todo código produzido em uma linguagem de programação é compilado por uma
ferramenta e, então, carrega-se os arquivos binários para o ESP8266 através da
serial, para que a execução do código seja iniciada. Na
\autoref{fig:esp-toolchain} é apresentado um modelo de desenvolvimento e
implantação desde o código até chegar no módulo ESP8266 e, na \autoref{table:tools-esp},
são apresentadas as ferramentas utilizadas como compiladores e carregadores.

\begin{figure}[htb]
	\caption{\label{fig:esp-toolchain}Sequência de ferramentas para implantação}
	\begin{center}
		\begin{verbatim}
		+--------+           +-----------+
		| Código |-----------> Compilador|
		+--------+           |           |      +----------------------+
		                     |  Ligador  <------| Cabeçalhos Espressif |
		                     +-----------+      +----------------------+
		                           |
		+------------+        +----v-----+
		| Carregador <--------+ Binários |
		+-----+------+        +----------+
		       \
		        \---Serial
		         \
		+---------v----+
		|  Módulo ESP  |
		+--------------+
		\end{verbatim}
	\end{center}
	\legend{Fonte: Elaborada pelo autor}
\end{figure}

\begin{figure}[htb]
	\caption{\label{fig:esp-pilha-serial}ESP-12f com regulador tensão e serial}
	\begin{center}
		\includegraphics[width=1\textwidth]{040-plataformas/esp-dev/breadboard.jpg}
	\end{center}
	\legend{Fonte: Elaborada pelo autor}
\end{figure}

\begin{table}[htb]
\IBGEtab{%
\ABNTEXchapterfont {
	\caption{Ferramentas para desenvolvimento com ESP8266}%
	\label{table:tools-esp}
}
}{%
\begin{tabular}{cccc}
	\toprule
	Ferramenta			&	Editor	&	Compilador e Ligador	&	Carregador	\\
	\midrule \midrule
	Arduino IDE			&	Sim		&	arduino C				&	Sim, mas não carrega binários pré compilados	\\
	\midrule
						&			&	NodeMCU Lua,			&		\\
	ESPlorer			&	Sim		&	MicroPython,			&	Não, conta com firmware específico	\\
						&			&	AT e RN2483				&		\\
	\midrule
	esptool.py			&	Não		&	Não 					&	Somente binários pré compilados	\\
	\midrule
	ESP8266 Flash		&			&							&		\\
	Downloader			&	Não		&	Não 					&	Somente binários pré compilados	\\
	\midrule
	NodeMCU Firmware	&			&							&		\\
	Programmer			&	Não		&	Não 					&	Somente binários pré compilados	\\
	\midrule
	\bottomrule
\end{tabular}%
}{%
	\fonte{Produzido pelo autor.}%
}
\end{table}

Todo código produzido é carregado para o módulo ESP8266 através de seu barramento
serial. Alguns modelos, como o LoLin e D1 mini, já apresentam
conversor serial para micro-USB. Para os que não possuem tal interface é
necessário utilizar um conversor serial-USB externo, a
\autoref{fig:esp-pilha-serial} demonstra esse método. As GPIOs do ESP-12f
são acessadas somente através de placas
de circuito impresso, então uma foi adquirida para a programação do mesmo.

Dos conversores serial-USB adquiridos, o modelo CH340G} não funcionou por
não ter driver compatível com o Windows 10, em contraste com o modelo
CP2102 que funcionou no mesmo sistema operacional.


\subsection{Testes e resultados - ESP8266}
\label{subsec:testes-esp}

O primeiro objetivo durante a programação dos módulos ESP8266 foi cumprir a
premissa  estabelecida no início deste capítulo de acessar o Modo Promíscuo da
interface Wi-Fi. Neste caso, procurou-se pelo ponto da API de
hardware do ESP8266 onde os pacotes destinados a outros dispositivos são
descartados, desativar este filtro, capturar e avaliar o pacote para localizar o
seu emissor.

A princípio, com o firmware AT, que é o padrão do módulo ESP-01, e com o
emulador de serial da Arduino IDE ou a aplicação Cool Term, é
possível configurar e utilizar o módulo por completo apenas com instruções AT
enviadas através da conexão serial. A primeira investigação sobre a API do protocolo
AT indicou \citeonline{room15} como uma fonte sucinta da documentação
oficial fornecida por \citeonline{espressifATwiki} do firmware AT e
não revelou nenhuma capacidade de ativar o Modo Promíscuo.

Também utilizou-se a linguagem C que foi compilada na Arduino IDE e
enviada ao ESP8266 com a extenção esp8266 by ESP8266 Community que inclui
os cabeçalhos de funções para que o compilador padrão da Arduino IDE gere
código  executável pelo ESP8266. Mesmo nesta API, nenhuma capacidade de ativar o
Modo Promíscuo foi encontrada.

\begin{figure}[htb]
	\caption{\label{fig:esp-arduino}Código em C compilado e implantado em um ESP8266}
	\begin{center}
		\includegraphics[width=1\textwidth]{040-plataformas/esp-dev/arduino-ide.png}
	\end{center}
	\nota[Esquerda]{Comandos AT no emulador de serial da \emph{Arduino IDE}.}%
	\nota[Direita]{Editor da \emph{Arduino IDE} com código C.}%
	\nota[Abaixo em preto]{Processo de \emph{upload} do firmware escrito em C.}%
	\fonte{Elaborado pelo autor.}%
\end{figure}


Uma nova tentativa para a programação  dos módulos escolhidos foi feita através de
\emph{toolchains} (conjunto de ferramentas para desenvolvimento de software) da
empresa Espressif e de um usuário do Github, muito utilizado para
projetos de ESP8266, \citeonline{Pfalcon}. Ambas as \emph{toolchains}
são SDKs de código aberto. Os \emph{scripts} foram feitos na linguagem C,
compilados nessas SDKs e transferidos para os módulos ESP8266. Neste caso,
a configuração delas mostrou-se um desafio, pois requisitavam uma versão
específica do \emph{Ubuntu Linux} que o computador pessoal utilizado para o desenvolvimento
não suporta. Também foi testada a utilização de máquinas virtuais mas, novamente,
a máquina do desenvolvedor não possui virtualização, impossibilitando esta opção.

Em conclusão, apesar do baixo custo e da documentação da comunidade aberta, o
ESP8266 não foi adotado como sensor, pois não foi possível colocá-lo em modo
prosmícuo, essencial para detectar pacotes entre dispositivo e os pontos de
acesso, inviabilizando completamente o uso desta plataforma mesmo sendo a
mais adequada e promissora no ponto de vista da construção de um produto final
por seu extremo baixo custo.


\section{Raspberry Pi}
\label{sec:Raspberry-Pi}

O \emph{Raspberry PI 3 Model B} é um computador \emph{single-board}  (única
placa) que tem o tamanho próximo ao de um cartão de crédito. Foi desenvolvido
pela \emph{Raspberry Pi Foundation} para promover o ensino da computação nas escolas.
Este computador possui:


\begin{alineas}
	\item 1 GB RAM;

	\item Processador Gráfico \emph{VideoCore IV 3D};

	\item ARM CPU de 1.2 GHz quad-core 64-bit.

	\item 4 portas USB;

	\item 40 pinos GPIOs;

	\item Porta HDMI;

	\item Porta \emph{Megabit Ethernet};

	\item Saída de aúdio e vídeo 3.5 mm;

	\item Interface para câmera (CSI) e monitor (DSI);

	\item Leitor para cartão \emph{micro SD};

	\item \emph{Wi-Fi LAN} embutida 802.11n;

	\item \emph{Bluetooth 4.1} e \emph{Bluetooth Low Energy} (BLE);

\end{alineas}

\begin{figure}[htb]
	\caption{\label{fig:rpi-3}Raspiberry Pi 3 }
	\begin{center}
		\includegraphics[width=1\textwidth]{040-plataformas/RPi-Wi-Fi-dongles/rpi-onboard.jpg}
	\end{center}
	\legend{Fonte: Elaborada pelo autor}
\end{figure}

Após o desenvolvimento e testes com ESP8266, foi testado e desenvolvido
software para transformar o Raspberry Pi em uma plataforma para hospedar o
sensor. Sua principal diferença é o sistema operacional linux (inexistente no
ESP8266) que favorece o Raspberry e o alto custo que o desfavorece. Em média no
exterior o Raspberry Pi é vendido por USD \$ 35,00 \cite{RPI2016}
e no Brasil entre R\$ 270 em Março de 2016 e R\$ 190 em Janeiro de 2017 \cite{rpi3-mercadolivre}.

As vantagens de ter um computador moderno completo sobrepõem seu custo em muitas
vezes. Dentre as quais destacamos a interface "amigável" com usuário devido ao
sistema operacional oferecendo maior nível de abstração (bastando apenas alguns
comandos para acessá-los realizar tarefas complexas), comunidade \emph{open
source} e o poder computacional.
Além deste recurso a nível de sistema, a comunidade e número de projetos "faça
você mesmo" é muito maior que a do ESP8266, devido a sua simplicidade em
conectar-se a um monitor e construir protótipos e aplicações.


\subsection{Disponibilidade no mercado}
\label{subsec:mercado-esp}



O Raspberry é ligado por uma fonte de 2A, 5V e 10W através de uma entrada micro
USB. Para ligá-la, foi adquirido uma fonte USB tipo A para iPad, pois além de
poder desconectar o cabo da fonte, facilitando a manutenção, fornece a
quantidade exata de amperagem que o computador precisa. A primeira aquisição foi
de um carregador de \emph{smartphone} que não forneceu os amperes necessários.

Figura x.x - Carregador USB
![](carregador-ipad.jpg)


**Sistema Operacional**

O Raspberry Pi comporta diversos sistemas operacionais que são carregados de seu
cartão \emph{microSD}. Alguns exemplos de sistemas compatíveis:
\emph{Archlinux}, \emph{OpenELECE}, \emph{Raspbian}, \emph{Risc OS},
\emph{Pidora}, \emph{Kali Linux}, \emph{Windows 10 IoT}, entre outros. Para este
trabalho, foi utilizado o \emph{Raspbian} cuja interface é mostrada na
\selfref{fig:raspbian-jessie}.

\begin{figure}[htb]
	\caption{\label{fig:raspbian-jessie}Raspbian Jessie com Pixel}
	\begin{center}
		\includegraphics[width=1\textwidth]{040-plataformas/RPi-Wi-Fi-dongles/raspbian.jpg}
	\end{center}
	\legend{Fonte: Elaborada pelo autor}
\end{figure}



\subsection{Desenvolvimento e Implantação}
\label{subsec:dev-esp}


\subsection{Testes e resultados}
\label{subsec:mercado-esp}



**Conclusão sobre Raspberry Pi**

O Raspberry foi adotado como o sensor para detectar os dispositivos. O modo
promíscuo conseguiu ser acessado através de adaptador/módulo USB Wi-Fi. Mais
detalhes sobre a construção e adoção deste computador serão apresentados no
capítulo "Construção".

**Comparativo RPi X ESP8266**

Em comparação com o ESP8266, o Raspberry Pi compensou seu preço mais caro devido
a facilidade de programação e acesso aos seus recursos e integração e acesso a
recursos externos. Além disso, foi possível chegar ao modo promíscuo facilmente
através do Bash e do sistema operacional. A seguir, uma tabela comparando as
principais características do RPi e do módulo ESP12F.

Figura X.X - RPi x ESP12F
![](rpi-esp.png)
Fonte: Elaborada pelo autor.


\chapter{Construção}
\label{chap:Construcao}

Para construção do \emph{software} aplicativo foi utilizado uma arquitetura em
três camadas: sensor, distribuidor de acesso (\emph{IoT gateway}) e apresentação
(\emph{Web}). Nesta divisão os sensores capturam as informações dos dispositivos
e repassam para a camada seguinte, no \emph{gateway} todas as partes se
encontram para fornecer e solicitar informações e, por último a camada de
apresentação coleta o que é enviado dos sensores e gera uma página \emph{Web}
para visualização dos dados capturados.

Esta divisão está de acordo com o padrão encontrado em outras aplicações
\emph{IoT} onde a última camada usualmente varia entre apresentação e mineração
de dados (\emph{Data Mining}).

A camada de sensor utilizou as tecnologias \emph{Node.js}, \emph{TShark} parte
do \emph{Wireshark} e \emph{MQTT.js}. A camada \emph{gateway} foi composta
basicamente pelo \emph{MQTT Broker} \emph{Mosquitto}. Por fim a camada de
apresentação utlizou as tecnologias \emph{Node.js}, \emph{MQTT.js}, \emph{html},
\emph{css}, \emph{javascript}, \emph{Bootstrap} e \emph{Google Maps API}.

\begin{figure}[htb]
	\caption{\label{fig-arq-app}Arquitetura da aplicação}
	\begin{center}
		\includegraphics[width=1\textwidth]{050-construcao/esquema-proj.png}
	\end{center}
	\legend{Fonte: Elaborada pelo autor}
\end{figure}


\section{Sensor}
\label{sec:app-sensor}


A aplicação sensor tem como objetivo capturar, avaliar e classificar pacotes de
Wi-Fi, inferir estatísticas de dispositivos e fornecer estas informações para
os interessados através do \emph{gateway}.

Para fazer a caputra dos pacotes na aplicação final, diferente do que foi
demonstrado na \autoref{subsec:testes-rpi}, em especial o \emph{airodump-ng} e
sua interface demonstrada na \autoref{fig-airodump}, foi utilizado o programa
\emph{TShark} cujo modo de operação serve melhor para a construção dos
\emph{Streams} que serão abodados em breve.

\begin{citacao}

	TShark é uma versão orientada ao terminal do Wireshark projetada para capturar
	e exibir pacotes quando uma interface de usuário interativa não é necessária ou
	disponível. Ele suporta as mesmas opções como wireshark. \

	\citeonline{tshark} Tradução Nossa.
\end{citacao}

Como foi estabelecido no capítulo anterior, \emph{TShark}  utiliza a saída
padrão  do terminal (\emph{stdout}) como sua saída principal, esta
característica foi explorada com aplicação \emph{nodejs}. Mais especificamente
com módulo \emph{child\_process},  que provê uma API que permite a criação e
controle de processos filhos do processo \emph{nodejs}.

\begin{citacao}

	Node.js é uma estrutura em tempo de execução construida sobre o motor de
	execução JavaScript V8 do Chrome. Node.js utiliza um modelo orientado a
	evento, de entrada e saída não bloqueante que o faz leve e eficiente.
	O ecosistema de pacotes do Node.js, npm, é o maior ecosistema de bibliotecas
	de código livre no mundo. \

	\citeonline{nodejs} Tradução Nossa.
\end{citacao}

Como também foi estabelecido anterioremte o \emph{TShark} é executado com o
comando e argumentos como mostrado à seguir, a diferença em relação aos testes e
na escolha da plataforma é a forma de execução, na maneira mostrada o processo é
criado utilizando o módulo \emph{child\_process} \cite{child_process} e os
argumentos são passados como um vetor (\emph{Array}).

\begin{lstlisting}[language=java]
const spawn = require('child_process').spawn;
const tsharkProcessoFilho = spawn(
	'tshark', [
		'-I',
		'-i', childIface,
		'-T', 'fields',
		'-E', 'separator=,',
		'-E', 'quote=d',
		'-e', 'wlan.sa',
		'-e', 'wlan.sa_resolved',
		'-e', 'wlan.ta',
		'-e', 'wlan.ta_resolved',
		'-e', 'radiotap.dbm_antsignal',
		'-e', 'wlan_mgt.ssid',
		'-Y', 'wlan.sa'
	]);
\end{lstlisting}

Para utilizar o resultado gerado pelo \emph{TShark} utilizamos outro método do
módulo \emph{child\_process} juntamente com a estrutura de \emph{Stream}
\cite{stream} que provê o método \emph{pipe(destination[, options])} que permite,
de maneira análoga ao operador '|' no terminal também chamado de \emph{pipe},
redirecionar a saída de um processo ou \emph{stream} de leitura para outro
processo ou \emph{stream} de escrita.

Com isso nos falta apenas um \emph{stream} de escrita que receba a saída do
\emph{TShark} que foi definida no formato CSV. Para isso biblioteca extra
\emph{fast-csv} deve ser instalada. Com ela podemos criar o \emph{stream}
necessário e configura-lo interpretar os resultados \cite{fast-csv}.

\begin{lstlisting}[language=java]
const csv = require("fast-csv");
let csvStream = csv()
  .on("data", function(data){
    let packet = new Packet(
      data[0], // sender address
      data[1], // sender address resolved
      data[2], // transmitter address
      data[3], // transmitter address resolved
      data[4], // potencia de sinal (rss)
      data[5], // nome da rede no pacote Beacon
    );
    processarPacote(packet);
  })
  .on("end", function(){
    console.log("done with tshark");
  });

tsharkProcessoFilho.stdout.setEncoding('utf8');
tsharkProcessoFilho.stdout.pipe(csvStream);
\end{lstlisting}

Nesta ultima linha pode-se observar a operação de redirecionamento (\emph{pipe})
de \emph{stream} da saída padrão do processo filho (\emph{stdout}) para o
\emph{stream} de escrita descrito e configurado com o \emph{fast-csv}. A parte
faltante é o processamento dos pacotes e envio das inferências do sensor para o
\emph{gateway}.

Para as inferências e estatísticas foi criada uma classe onde as informações
sobre cada dispositivo são agregadas e um objeto para armazenamento indexado
pelo endereço MAC. Desta forma cada novo pacote pode ser acrescentado ao
histórico de cada dispositivo.

\begin{lstlisting}[language=java]
let devices = {};	// lista indexada de dispositivos

class Device {
  [...]
  appendPacket(packet){
    let curTime = ( new Date() ).toISOString();
    this.rssHistory.push(packet.radiotap.dbm_antsignal);
    this.ssidHistory[packet.wlan_mgt.ssid] = curTime;
    this.taHistory[packet.wlan.ta] = {
      ta      : packet.wlan.ta,
      ta_resolved  : packet.wlan.ta_resolved,
    };
  }
}

function processarPacote(packet) {
  let sa = packet.wlan.sa;
  if (! devices[sa]){
    devices[sa] = new Device(sa, packet.wlan.sa_resolved);
  }
  devices[sa].appendPacket(packet);
}
\end{lstlisting}

A partir desta lista é posível calcular a média e desvio padrão de potência de
sinal para cada dispositivo descoberto.


\begin{lstlisting}[language=java]
class Device {
  [...]
  get rssStatistics(){
    let sum = 0;
    let avg = 0;
    let variance = 0;
    let stdDeviation = 0;
    if (this.rssHistory.length > 0){
      for (let rss of this.rssHistory){
        sum += rss;
      }
      avg = sum / this.rssHistory.length;

      sum = 0;
      for (let rss of this.rssHistory){
        sum += Math.pow(( rss - avg ), 2);
      }
      variance = sum / (this.rssHistory.length - 1);
      stdDeviation = Math.sqrt(variance);
    }
    return {
      size      : this.rssHistory.length,
      avg        : avg,
      variance    : variance,
      stdDeviation  : stdDeviation,
      time      : ( new Date() ).toISOString(),
    };
  }
}
\end{lstlisting}

Por fim, é necessário comunicar aos interessados nessas inferências, o que é
feito através do módulo \citeonline{mqttjs}.

\begin{lstlisting}[language=java]
const mqtt = require('mqtt');
let clientMqtt  = mqtt.connect(`mqtt://${config.mqttHost}:${config.mqttPort}`, {
  username  : config.mqttUser,
  password  : config.mqttPwd,
})
clientMqtt.on('connect', function () {
  clientMqtt.subscribe('devices');
  startTshark();
});
clientMqtt.on('message', function (topic, message) {
  if (topic.toString() == 'devices') {
    switch (message.toString()) {
    case 'list':
      clientMqtt.publish(
        'devices/report',
        JSON.stringify( tshark.getDevices() )
      );
      break;
    case 'report':
      clientMqtt.publish(
        'devices/report',
        JSON.stringify( tshark.getReport() )
      );
      break;
    default:
      let mac = message.toString();
      clientMqtt.publish(
        'devices/report',
        JSON.stringify( tshark.getDeviceReport(mac) )
      );
      break;
    }
  }
});
\end{lstlisting}

Neste caso, quando a aplicação é iniciada ela conecta-se ao \emph{MQTT Broker},
quando a conexão é estabelecida ela solicita ao \emph{MQTT Broker} a inscrição
para receber as publicações no tópico 'devices' e o processo filho \emph{TShark}
é iniciado. O processo filho é executado, os resultados são capturados e
guardados.

Quando uma mensagem no tópico 'devices' é recebida três reações podem acontecer:
A lista de dispositivos deve ser fornecida, o relatório do sensor deve ser
fornecido ou um relatório sobre um dispositivo específico deve ser fornecido.
Para todos os casos uma resposta adequada é imediatamente processada e enviada
para o tópico 'devices/report'.

Em conclusão, a aplicação sensor instancia o processo de aquisição de dados
\emph{TShark} assincronamente e captura, classifica e armazena todos os pacotes
que ficam acessíveis através de requisições ao tópico \emph{MQTT} 'devices'.


\section{Gateway}
\label{sec:app-gw}

\begin{citacao}

	Eclipse Mosquitto™ is an open source (EPL/EDL licensed) message broker that
	implements the MQTT protocol versions 3.1 and 3.1.1. MQTT provides a lightweight
	method of carrying out messaging using a publish/subscribe model. This makes it
	suitable for "Internet of Things" messaging such as with low power sensors or
	mobile devices such as phones, embedded computers or microcontrollers like the
	Arduino. \

	\citeonline{nodejs} Tradução Nossa.
\end{citacao}


\section{Apresentação Web}
\label{sec:app-web}


\begin{figure}[htb]
	\caption{\label{fig-web-app}Web APP}
	\begin{center}
		\includegraphics[width=1\textwidth]{050-construcao/web-app.png}
	\end{center}
	\legend{Fonte: Elaborada pelo autor}
\end{figure}

\section{Sensor}
\label{sec:app-sensor}


A aplicação sensor tem como requisitos funcionais capturar, avaliar e classificar pacotes de
\emph{Wi-Fi}, inferir estatísticas de dispositivos e fornecer estas informações para
os interessados através do \emph{gateway}.

Para fazer a captura dos pacotes na aplicação final, diferente do que foi
demonstrado na \autoref{subsec:testes-rpi}, em especial o \emph{airodump-ng} e
sua interface demonstrada na \autoref{fig-airodump}, foi utilizado o programa
\emph{TShark} cujo modo de operação serve melhor para a construção dos
\emph{Streams} que serão abordados em breve.

\begin{citacao}

	TShark é uma versão orientada ao terminal do Wireshark projetada para capturar
	e exibir pacotes quando uma interface de usuário interativa não é necessária ou
	disponível. Ele suporta as mesmas opções como wireshark. \

	\citeonline{tshark} Tradução Nossa.
\end{citacao}

Como foi estabelecido no capítulo anterior, \emph{TShark}	utiliza a saída
padrão	do terminal (\emph{stdout}) como sua saída principal, esta
característica foi explorada com a aplicação \emph{nodejs}. Mais especificamente,
com o módulo \emph{child\_process},	que provê uma API que permite a criação e
controle de processos filhos do processo \emph{nodejs}.

\begin{citacao}

	Node.js é uma estrutura em tempo de execução construida sobre o motor de
	execução JavaScript V8 do Chrome. Node.js utiliza um modelo orientado a
	evento, de entrada e saída não bloqueante que o faz leve e eficiente.
	O ecosistema de pacotes do Node.js, npm, é o maior ecossistema de bibliotecas
	de código livre no mundo. \

	\citeonline{nodejs} Tradução Nossa.
\end{citacao}

Como também foi estabelecido anteriormente, o \emph{TShark} é executado com o
comando e argumentos como mostrado a seguir, a diferença em relação aos testes e
na escolha da plataforma é a forma de execução, na maneira mostrada, o processo é
criado utilizando o módulo \emph{child\_process} \cite{child_process} e os
argumentos são passados como um vetor (\emph{Array}).

\begin{lstlisting}[language=javascript,caption={TShark e opções executado pelo Node.js},label=code-node-tshark]
const spawn = require('child_process').spawn;
const tsharkProcessoFilho = spawn(
	'tshark', [
		'-I',
		'-i', childIface,
		'-T', 'fields',
		'-E', 'separator=,',
		'-E', 'quote=d',
		'-e', 'wlan.sa',
		'-e', 'wlan.sa_resolved',
		'-e', 'wlan.ta',
		'-e', 'wlan.ta_resolved',
		'-e', 'radiotap.dbm_antsignal',
		'-e', 'wlan_mgt.ssid',
		'-Y', 'wlan.sa'
	]);
\end{lstlisting}

Para utilizar o resultado gerado pelo \emph{TShark}, utilizou-se outro método do
módulo \emph{child\_process} juntamente com a estrutura de \emph{Stream}
\cite{stream} que provê o método \emph{pipe(destination[, options])} que permite,
de maneira análoga ao operador '|' no terminal também chamado de \emph{pipe},
redirecionar a saída de um processo ou \emph{stream} de leitura para outro
processo ou \emph{stream} de escrita.

Com isso, ainda falta um \emph{stream} de escrita que receba a saída do
\emph{TShark} que foi definida no formato CSV. Para isso, uma biblioteca extra,
\emph{fast-csv}, deve ser instalada. Com ela, pode-se criar o \emph{stream}
necessário e configurá-lo para interpretar os resultados \cite{fast-csv}.

\begin{lstlisting}[language=javascript,caption={Uso do fast-csv},label=code-node-csv]
const csv = require("fast-csv");
let csvStream = csv()
	.on("data", function(data){
		let packet = new Packet(
			data[0], // sender address
			data[1], // sender address resolved
			data[2], // transmitter address
			data[3], // transmitter address resolved
			data[4], // potencia de sinal (rss)
			data[5] // nome da rede no pacote Beacon
		);
		processarPacote(packet);
	})
	.on("end", function(){
		console.log("done with tshark");
	});
tsharkProcessoFilho.stdout.setEncoding('utf8');
tsharkProcessoFilho.stdout.pipe(csvStream);
\end{lstlisting}

Na linha 18, pode-se observar a operação de redirecionamento (\emph{pipe})
de \emph{stream} da saída padrão do processo filho (\emph{stdout}) para o
\emph{stream} de escrita descrito e configurado com o \emph{fast-csv}. A parte
faltante é o processamento dos pacotes e envio das inferências do sensor para o
\emph{gateway}.

Para as inferências e estatísticas, foi criado um objeto para armazenamento indexado
pelo endereço MAC e uma classe onde as informações
sobre cada dispositivo são agregadas. Desta forma, cada novo pacote pode ser acrescentado ao
histórico de cada dispositivo.


\begin{lstlisting}[language=javascript,caption={Adição do pacote ao histórico do dispositivo},label=code-device-packet]
let devices = {};	// lista indexada de dispositivos
class Device {
	[...]
	appendPacket(packet){
		let curTime = ( new Date() ).toISOString();
		this.rssHistory.push(packet.radiotap.dbm_antsignal);
		this.ssidHistory[packet.wlan_mgt.ssid] = curTime;
		this.taHistory[packet.wlan.ta] = {
			ta			: packet.wlan.ta,
			ta_resolved	: packet.wlan.ta_resolved,
		};
	}
}
function processarPacote(packet) {
	let sa = packet.wlan.sa;
	if (! devices[sa]){
		devices[sa] = new Device(sa, packet.wlan.sa_resolved);
	}
	devices[sa].appendPacket(packet);
}
\end{lstlisting}

A partir desta lista, é posível calcular a média e o desvio padrão de potência de
sinal para cada dispositivo descoberto.

\begin{lstlisting}[language=javascript,caption={Extração das estatísticas do dispositivo},label=code-device-stats]
class Device {
	[...]
	get rssStatistics(){
		let sum = 0, avg = 0, variance = 0, stdDeviation = 0;
		if (this.rssHistory.length > 0){
			for (let rss of this.rssHistory){
				sum += rss;
			}
			avg = sum / this.rssHistory.length;
			sum = 0;
			for (let rss of this.rssHistory){
				sum += Math.pow(( rss - avg ), 2);
			}
			variance = sum / (this.rssHistory.length - 1);
			stdDeviation = Math.sqrt(variance);
		}
		return {
			size			: this.rssHistory.length,
			avg				: avg,
			stdDeviation	: stdDeviation,
		};
	}
}
\end{lstlisting}


Por fim, é necessário comunicar aos interessados nessas inferências, o que é
feito através do módulo \citeonline{mqttjs}.

\begin{lstlisting}[language=javascript,caption={Cliente MQTT.js},label=code-mqttjs-client]
const mqtt = require('mqtt');
const mqttBrok = `mqtt://${config.mqttHost}:${config.mqttPort}`;
let clientMqtt	= mqtt.connect(mqttBrok, {
	username	: config.mqttUser,
	password	: config.mqttPwd,
})
clientMqtt.on('connect', function () {
	clientMqtt.subscribe('devices');
	startTshark();
});
clientMqtt.on('message', function (topic, message) {
	if (topic.toString() == 'devices') {
		switch (message.toString()) {
		case 'list':
			clientMqtt.publish(
				'devices/report',
				JSON.stringify( tshark.getDevices() )
			);
			break;
		case 'report':
			clientMqtt.publish(
				'devices/report',
				JSON.stringify( tshark.getReport() )
			);
			break;
		default:
			let mac = message.toString();
			clientMqtt.publish(
				'devices/report',
				JSON.stringify( tshark.getDeviceReport(mac) )
			);
			break;
		}
	}
});
\end{lstlisting}

No caso desta listagem de código, quando a aplicação é iniciada ela conecta-se ao \emph{MQTT Broker}.
Quando a conexão é estabelecida, ela solicita ao \emph{MQTT Broker} a inscrição
para receber as publicações no tópico 'devices' e o processo filho \emph{TShark}
é iniciado. O processo filho é executado, os resultados são capturados e
guardados.

Quando uma mensagem no tópico 'devices' é recebida, três reações podem acontecer:
a lista de dispositivos deve ser fornecida, o relatório do sensor deve ser
fornecido ou um relatório sobre um dispositivo específico deve ser fornecido.
Para todos os casos, uma resposta adequada é imediatamente processada e enviada
para o tópico 'devices/report'.

Em conclusão, a aplicação sensor instancia o processo de aquisição de dados
\emph{TShark} assincronamente e captura, classifica e armazena todos os pacotes
que ficam acessíveis através de requisições ao tópico \emph{MQTT} 'devices'.
Isso cobre os requisitos funcionais desta aplicação.

Os requisitos não-funcionais desta aplicação estão ligados com o ambiente de
implantação que é um RPI3 remoto, sem outro dispositivo de entrada e saída
exceto a comunicação \emph{MQTT}, portanto, sem nenhum aspecto permitindo
monitoração de um humano que garanta que o aplicativo continue funcionando.

Este ambiente se assemelha muito a aplicações em núvem, onde não há acesso
físico ao computador onde a aplicação é executada. Neste ambiente, procura-se
garantir que a aplicação seja executada constantemente e atualizada
automaticamente sempre que uma nova versão é construída e testada. Estas
garantias podem ser alcançadas com ferramentas de integração contínua
(\emph{Continuous Integration} - CI). No caso do Github, a escolha de hospedagem
de código deste projeto, diversas opções deste tipo de ferramenta podem ser
encontradas em \citeonline{githubdeploy}.

No entanto, para esta aplicação, construiu-se uma solução extremamente
simplificada destas ferramentas utilizando a arquitetura existente do programa
\emph{git} para garantir que novas versões fossem instaladas assim que possível.
A implementação consiste da adição, nas primeiras instruções do aplicativo,
uma rotina que executa sincronamente um programa externo através do terminal
(diferente da execução do \emph{TShark} que é assíncrona). Este programa é
o \emph{git} com a opção \emph{pull} que, quando executado em um diretório que
é repositório, solicita ao servidor configurado, como \emph{origin}, a atualização
do código do \emph{branch} atual. Se a mensagem encontrada na saída padrão
(\emph{stdout}) resultante deste comando for diferente de
\emph{"Already up-to-date."}, o programa finaliza imediatamente, pois uma nova
versão foi baixada e uma reinicialização da aplicação é necessária.
Em resumo, uma vez instalada a aplicação utilizando a ferramenta \emph{git}, ela
será atualizada com o mesmo processo com que foi instalada.

O outro requisito é que a aplicação seja executada constantemente. Para isso,
pode-se utilizar uma aplicação também escrita em \emph{Node.js} e disponível no
gerenciador de pacotes \emph{npm}: \emph{forever-service}. Ela registra um novo serviço no sistema operacional
linux para que a infra-estrutura de gerenciamento de serviços	existente seja
aproveitada (por exemplo: início automático após o início do sistema). Além disso, a sua dependência
\emph{forever} garante que a aplicação seja reinicializada sempre que finaliza,
executando a aplicação constantemente \cite{forever-service}.

\section{Gateway}
\label{sec:app-gw}

Para a construção do \emph{Gateway} foi feita a instalação e a configuração do
\emph{MQTT Broker} \emph{Mosquitto} em um dos RPI3.

\begin{citacao}

	Eclipse Mosquitto™ é um distribuidor de mensagens de código aberto (EPL/EDL
	licenciado) que implementa o protocolo MQTT versões 3.1 e 3.1.1. O MQTT
	fornece um método leve de transmitir mensagens usando um modelo de
	publicação/inscrição. Isso o torna adequado para mensagens "Internet das
	Coisas", como com sensores de baixa potência ou dispositivos móveis, como
	telefones, computadores embutidos ou microcontroladores como o Arduino. \

	\citeonline{mosquitto} Tradução Nossa.
\end{citacao}


A instalação é realizada com o gerenciador de pacotes \emph{apt-get} padrão do
\emph{raspbian} como mostrado na primeira linha da listagem abaixo.

\begin{lstlisting}[language=bash]
pi@broker:~ $ sudo apt-get install mosquitto
pi@broker:~ $ sudo sh -c 'echo "password_file /etc/mosquitto/passwd"
	>> /etc/mosquitto/mosquitto.conf'
pi@broker:~ $ sudo mosquitto_passwd -c /etc/mosquitto/passwd user
Password:
Reenter password:
\end{lstlisting}

Nas linhas 2 e 3, adiciona-se ao arquivo padrão de configuração padrão do
\emph{Mosquitto} a linha 'password\_file /etc/mosquitto/passwd' que
indica o arquivo de senhas a ser utilizado para autenticação. Na linha 4, é
adicionado o usuário 'user' ao arquivo, a senha é solicitada  nas
linhas 5 e 6.

Feito isso, o acesso ao \emph{Broker} está limitado aos usuários e senhas
configurados neste processo. As demais configurações permaneceram sem alterações.

Na configuração do sensor, deve ser adicionado o endereço (nome ou IP), a porta
\emph{1883} e o par usuário e senha para que o sensor acesse o \emph{Gateway} com
sucesso.

Para verificar o funcionamento do conjunto sensor e distribuidor (\emph{Gateway}), utiliza-se um
cliente \emph{MQTT}, como o aplicativo para a plataforma \emph{Android} \emph{MQTT Dashboard}
\cite{mqttdash} ou os aplicativos para as plataformas \emph{Java} \emph{mqtt-spy}
\cite{mqttspy} e \emph{Windows} \emph{MQTT.fx} \cite{mqttfx}.

Nas figuras \ref{fig-ad-publish}, \ref{fig-ad-home-ack} e \ref{fig-ad-admin-ack},
o aplicativo \emph{MQTT Dashboard} é utilizado para verificar quais sensores
estão online com a mensagem 'echo' publicada no tópico 'ADMIN'. Na
\autoref{fig-ad-publish}, os botões enviam mensagens com simplicidade. Na
\autoref{fig-ad-home-ack}, é visível a lista de tópicos e a última mensagem
recebida em cada um deles. Na \autoref{fig-ad-admin-ack}, fica evidente a
resposta de confirmação (\emph{ack}) de cada um dos sensores.

A aplicação \emph{MQTT.fx}, em especial a tela mostrada na
\autoref{fig-mqttfx-stats}, revela estatísticas do \emph{Broker}, como a versão,
tempo \emph{online}, número de clientes, mensagens e utilização de rede.
Na aplicação \emph{mqtt-spy}, o comando de listagem mencionado na
\autopageref{code-sensor-mqtt} é demonstrado.

\begin{figure}[htb]
\centering
	\begin{minipage}{0.32\textwidth}
	\centering
		\caption{\label{fig-ad-publish}MQTT Dashboard: Envio de mensagens}
		\includegraphics[width=1\textwidth]{052-gateway/mqtt/ad-publish.jpg}
		\legend{Fonte: Produzido pelo autor}
	\end{minipage}
\hfill
	\begin{minipage}{0.32\textwidth}
	\centering
		\caption{\label{fig-ad-home-ack}MQTT Dashboard: Lista de inscrições}
		\includegraphics[width=1\textwidth]{052-gateway/mqtt/ad-home-ack.jpg}
		\legend{Fonte: Produzido pelo autor}
	\end{minipage}
\hfill
	\begin{minipage}{0.32\textwidth}
	\centering
		\caption{\label{fig-ad-admin-ack}MQTT Dashboard: Lista de mensagens no tópico}
		\includegraphics[width=1\textwidth]{052-gateway/mqtt/ad-admin-ack.jpg}
		\legend{Fonte: Produzido pelo autor}
	\end{minipage}
\end{figure}

\begin{figure}[htb]
	\centering
	\caption{\label{fig-mqttfx-stats}MQTT.fx: Estatísticas do \emph{Broker}}
	\includegraphics[width=1\textwidth]{052-gateway/mqtt/mqttfx-stats.png}
	\legend{Fonte: Produzido pelo autor}
\end{figure}

\begin{figure}[htb]
	\centering
	\caption{\label{fig-mqtt-spy-list}mqtt-spy: Listagem de dispositivos}
	\includegraphics[width=1\textwidth]{052-gateway/mqtt/mqtt-spy-list.png}
	\legend{Fonte: Produzido pelo autor}
\end{figure}

\section{Apresentação Web}
\label{sec:app-web}

Como já mencionado, para a apresentação das informações de captura de maneira
acessível, foi construída uma aplicação \emph{Web} utilizando as tecnologias
Node.js, MQTT.js, \emph{html}, \emph{css}, \emph{javascript},
\emph{Bootstrap} e \emph{Google Maps API}.

Node.js e a biblioteca de cliente MQTT.js foram utilizados para,
da mesma forma exposta no \autoref{code-mqttjs-client}, conectar
uma aplicação escrita em \emph{javascript} com o \emph{MQTT Broker}. Esta
aplicação recupera as informações sobre os dispositos descobertos e as
classifica por proximidade de cada sensor para compor a lista de dispositios
por sensor vista no lado direito da \autoref{fig-web-app}.

Já o \emph{html}, \emph{css}, \emph{javascript} e \emph{Bootstrap} foram
utilizados para estruturar, estilizar, inflar e animar as informações. Em
especial, o \emph{Bootstrap} forneceu a estrutura de cabeçalho, rodapé e colunas,
além do esquema de cores.

A \emph{Google Maps API} juntamente com um pouco de \emph{css} e
\emph{javascript} fornece o mapa visto no lado direito da \autoref{fig-web-app}.
Nele estão reprensentadas as localizações geográficas de cada sensor
representados pelos marcadores nas cores azul e verde.

\begin{figure}[htb]
	\caption{\label{fig-web-app}Web APP}
	\begin{center}
		\includegraphics[width=1\textwidth]{053-web/web-app.png}
	\end{center}
	\legend{Fonte: Elaborada pelo autor}
\end{figure}


\chapter{Resultados e Discussão}
\label{chap:resultados}

Neste capítulo, são abordados, analisados e discutidos os resultados encontrados
durante a exploração do tema e das plataformas e durante a implementação das
aplicações, além de verificar a precisão atingida com a aplicação implementada. Todos os testes foram realizados no prédio do laboratório
LTIA da Unesp de Bauru, onde os sensores permaneceram monitorando dispositivos.



\section{Método de teste}
\label{sec:metodo-teste}

Como discutido no \autoref{chap:Construcao}, a arquitetura geral da aplicação (\autoref{fig-arq-app})
mostra que a precisão vista na aplicação \emph{Web} depende dos resultados
encontrados pela aplicação sensor que, por sua vez, depende do par de
capacidades combinadas do \emph{hardware} adaptador Wi-Fi e do \emph{software}
\emph{TShark}. Portanto, a metodologia de testes empregada neste capítulo é
analisar diretamente as capacidades deste último par. Esta decisão também se
deve pela facilidade de armazenar e analisar arquivos CSV gerados pelo
\emph{TShark}.

As capturas foram executadas com o comando descrito no \autoref{code-tshark-pipe-assinc}
que é o mesmo utilizado na aplicação sensor.

\begin{lstlisting}[language=bash,caption={TShark e redirecionamento da saída para arquivo assíncrono},label=code-tshark-pipe-assinc]
pi@sensor-01:~ $ tshark -I -i wlan0 -T fields -E header=y -E quote=d \
-e wlan.sa -e wlan.sa_resolved -e wlan.ta -e wlan.ta_resolved \
-e radiotap.dbm_antsignal -e wlan_mgt.ssid \
>> 2017-01-17--02-48--rpi-02.csv &
pi@sensor-01:~ $ exit
\end{lstlisting}

Neste modo de uso, os resultados são direcionadas para a saída padrão
(stdout)  do terminal e podem ser capturados por outro programa no formato
de valores separados por vírgula (CSV). Os campos escolhidos para captura
são \emph{wlan.sa}, \emph{wlan.sa\_resolved}, \emph{wlan.ta},
\emph{wlan.ta\_resolved}, \emph{radiotap.dbm\_antsignal} e \emph{wlan\_mgt.ssid}.

Na linha 4 do \autoref{code-tshark-pipe-assinc}, o \emph{\&} representa o início
de um processo independente (assíncrono) e, a linha 5, a finalização do terminal.
Esta operação foi somente executada durante a captura longa.

A análise de dados foi feita com a função \emph{summary} da ferramenta
\emph{Ron’s editor}\footnote{\url{https://www.ronsplace.eu/Products/RonsEditor}}
e a filtragem com a função \emph{Filter} da ferramenta
\emph{RecCsvEditor}\footnote{\url{http://recsveditor.sourceforge.net/}}. Para a
construção dos gráficos, foi utilizada a ferramenta
\emph{WPS Spreadsheets}\footnote{\url{https://www.wps.com/office-free}}.

\section{Avaliação de ruído e consistência}
\label{sec:teste-ruido}

Para entender o ambiente onde a aplicação foi desenvolvida e testada no âmbito
de ruído e pontos de referência \emph{Wi-Fi}, foi executada uma captura de referência
durante a noite quando ninguém estava no prédio protótipo, ou seja, nenhum
dispositivo foi movimentado até a manhã seguinte.

Nesta captura, dois sensores foram posicionados a menos de 10
centímetros de distância um do outro sobre uma mesa a um metro do chão indicada no ponto
azul da \autoref{fig-planta-baixa-juntos}. A captura
ocorreu das \texttt{2:50} da noite até aproximadamente \texttt{11:25} da manhã, totalizando
aproximadamente \texttt{8} horas de captura.

\begin{figure}[htb]
	\caption{\label{fig-planta-baixa-juntos}Ambiente de teste de ruído}
	\begin{center}
		\includegraphics[width=1\textwidth]{060-testes/data-analisis/planta-baixa-ruido.png}
	\end{center}
	\nota[Em azul]{Sensores da aplicação}%
	\nota[Em verde]{Pontos de acesso da rede \emph{Wi-Fi} do laboratório}%
	\legend{Fonte: Elaborada pelo autor}
\end{figure}

As distâncias aproximadas, em linha reta, entre os sensores e o dispositivo
062722b3e5fe (AP na sala de servidores, anexo ao salão) é de
\texttt{14,6m} e entre os sensores e o dispositivo
062722b3e5fb (AP na sala de aula) é de \texttt{26,5m}.

Para o primeiro sensor, a função \emph{summary} da ferramenta \emph{Ron’s editor} indicou que foram capturados \texttt{1.729.624}
pacotes num arquivo de \texttt{155 MB} com \texttt{88} transmissores únicos.
Para o segundo sensor, a mesma função indicou que foram capturados \texttt{1.554.319}
pacotes num arquivo de \texttt{134 MB} com \texttt{66} transmissores únicos.
Em ambos os sensores, se destacaram dois endereços MAC que são os pontos de
acesso para rede \emph{Wi-Fi} do laboratório.

Os pacotes capturados dos pontos de acesso e os valores de potência de sinal
associados a eles são notáveis para entender a precisão de um sistema de localização
desta categoria uma vez que esses APs estão fixos e trasmitiram o
maior número de pacotes nesta captura.

Nas \autoref{fig-062722b3e5fb-s1}, \autoref{fig-062722b3e5fb-s2},
\autoref{fig-062722b3e5fe-s1} e \autoref{fig-062722b3e5fe-s2} podemos observar
a potência de sinal para cada pacote capturado em ordem de chegada. Na \autoref{fig-062722b3e5fb-s1} e
na \autoref{fig-062722b3e5fb-s2}, os pacotes
foram capturados, respectivamente, pelos sensores 1 e 2 para o ponto de acesso de MAC 062722b3e5fb. E na
\autoref{fig-062722b3e5fe-s1} e na \autoref{fig-062722b3e5fe-s2}, os pacotes foram capturados, respectivamente, pelos sensores 1 e 2 para
o ponto de acesso de MAC 062722b3e5fe.

\begin{figure}[htb]
	\begin{minipage}{0.49\textwidth}
	\centering
		\caption{\label{fig-062722b3e5fb-s1}Sinal em dBm por pacote capturado - 062722b3e5fb sensor 1}
		\includegraphics[width=1\textwidth]{060-testes/data-analisis/night-run/062722b3e5fb-sensor-01.png}
		\legend{Fonte: Elaborada pelo autor}
	\end{minipage}
\hfill
	\begin{minipage}{0.49\textwidth}
	\centering
		\caption{\label{fig-062722b3e5fb-s2}Sinal em dBm por pacote capturado - 062722b3e5fb sensor 2}
		\includegraphics[width=1\textwidth]{060-testes/data-analisis/night-run/062722b3e5fb-sensor-02.png}
		\legend{Fonte: Elaborada pelo autor}
	\end{minipage}
\hfill
	\begin{minipage}{0.49\textwidth}
	\centering
		\caption{\label{fig-062722b3e5fe-s1}Sinal em dBm por pacote capturado - 062722b3e5fe sensor 1}
		\includegraphics[width=1\textwidth]{060-testes/data-analisis/night-run/062722b3e5fe-sensor-01.png}
		\legend{Fonte: Elaborada pelo autor}
	\end{minipage}
\hfill
	\begin{minipage}{0.49\textwidth}
	\centering
		\caption{\label{fig-062722b3e5fe-s2}Sinal em dBm por pacote capturado - 062722b3e5fe sensor 2}
		\includegraphics[width=1\textwidth]{060-testes/data-analisis/night-run/062722b3e5fe-sensor-02.png}
		\legend{Fonte: Elaborada pelo autor}
	\end{minipage}
\end{figure}

Imediatamente, percebe-se que, apesar da distância ser invariável durate todo o
teste, a potência de sinal
não permaneceu constante. Nota-se também que o valor
nunca assume valor par. Estas duas características fazem com que o gráfico
seja feito de 7 linhas paralelas ao centro e alguns grupos de pontos ao redor.

Em todas os gráficos anteriores, nota-se uma abrupta mudança no comportamento próximo do pacote 225.000.
Estimou-se que esta mudança é devido ao horário (08:00) em que a universidade
(incluindo o prédio piloto e seus arredores) inicia o seu funcionamento.

Para ter uma visão geral clara da distribuição das potências de sinal, calculou-se
a média e o desvio padrão para cada um dos casos. Além disso, adicionou-se o
quanto o desvio padrão representa da média (erro).

\begin{table}[htb]
\IBGEtab{%
\ABNTEXchapterfont {
	\caption{\label{table:ap-pwr-avg}dBm Pontos de acesso - Acumulado 8 horas}%
}
}{%
\begin{tabular}{c|cc|cc}
\toprule
\midrule
AP							& \multicolumn{2}{c}{06:27:22:b3:e5:fe}		&	\multicolumn{2}{c}{06:27:22:b3:e5:fb}	\\
							&	Sensor 1		&	Sensor 2			&	Sensor 1		&	Sensor 2			\\
\midrule
Pacotes						&	300403			&	299864				&	305735			&	299264				\\
\midrule
Média (dBm)					& 	-57.66137822	&	-52.57644132		&	-73.04459417	&	-76.12900984		\\
\midrule
$\sigma$ (Desvio Padrão em dBm)	&	4.746974756		&	3.712407998			&	3.05045545		&	2.889560991			\\
$\sigma$ \%					&	8\%				&	7\%					&	4\%				&	4\%					\\
\midrule
%$3 \times \sigma $			&	14.24092427		&	11.13722399			&	9.151366351		&	8.668682972			\\
%$3 \times \sigma $ \%		&	25\%			&	21\%				&	13\%			&	11\%				\\
%\midrule
\bottomrule
\end{tabular}%
}{%
	\fonte{Produzido pelo autor.}%
}
\end{table}


Em alguns trabalhos, pode-se
encontrar a equação de FSPL (\emph{Free-space path loss} - perca no caminho em
espaço aberto) que é usualmente utilizada para determinar a potência do sinal
a ser transmitido para que este alcance o seu destinatário.

\begin{equation}
	{\mbox{FSPL(dB)}}=20\log _{{10}}(\text{d})+20\log _{{10}}(\text{f})+92.45
\end{equation}

Na aplicação \emph{Android} \emph{Wifi Distance Calculator}, desenvolvida por
\citeonline{Kuik2016}, e no trabalho de \citeonline{androidtri} a seguinte
equação é utilizada.

\begin{equation}
	\text{d} = 10 ^{ \frac{1}{20} \left( \text{p} - 20 \times log_{10}\left(\text{f}\right)  + 27.55 \right) } \\
\end{equation}

O $d$ é a distância em metros, $p$ é a potência do sinal em $dB$ e $f$
é a frequência do sinal em $MHz$. Como as redes \emph{Wi-Fi} utilizam canais na região
de 2.4GHz e 5GHz \cite{ieee80211}, pode-se simplificar a equação e chegar nos
resultados \autoref{eq:2.4ghz} e \autoref{eq:5ghz} respectivamente.

\begin{align}
\text{d}	&= 10 ^{ \frac{1}{20} \left( \text{p} - 20 \times log_{10}\left(\text{2400}\right)  + 27.55 \right) } \nonumber \\
			&= 10 ^{ \frac{1}{20} \left( \text{p} - 40.0542 \right) } \label{eq:2.4ghz}
\end{align}

\begin{align}
\text{d}	&= 10 ^{ \frac{1}{20} \left( \text{p} - 20 \times log_{10}\left(\text{5000}\right)  + 27.55 \right) } \nonumber \\
			&= 10 ^{ \frac{1}{20} \left( \text{p} - 46.4294 \right) } \label{eq:5ghz}
\end{align}

Utilizando a \autoref{eq:2.4ghz} (para 2.4GHz), pode-se inferir as distâncias entre os sensores
e pontos de acesso a partir da \autoref{table:ap-pwr-avg}. Na
\autoref{table:ap-distance}, estão presentes a distância em função da potência $d(p)$
calculada para a potência média ($\overline{dBm}$) e para o erro.


\begin{table}[htb]
\IBGEtab{%
\ABNTEXchapterfont {
	\caption{\label{table:ap-distance}Distância entre os sensores e os Pontos de acesso}%
}
}{%
\begin{tabular}{c|cc|cc}
\toprule
\midrule
AP							& \multicolumn{2}{c|}{06:27:22:b3:e5:fe}		&	\multicolumn{2}{c}{06:27:22:b3:e5:fb}	\\
							&	Sensor 1		&	Sensor 2			&	Sensor 1		&	Sensor 2			\\
\midrule
$d(\overline{dBm})$ metros
							&	7.639569932		&	4.254240777			&	44.89828049		&	64.03987741	\\
\midrule
$d(\overline{dBm} + \sigma)$ metros
							&	13.19525078		&	6.522926214			&	63.78998224		&	89.31585118	\\
$d(\overline{dBm} - \sigma)$ metros
							&	4.423032932		&	2.774608204			&	31.60144461		&	45.91688759	\\
\midrule
Erro acumulado metros		&					&						&					&				\\
$d(\overline{dBm}-\sigma)-d(\overline{dBm}+\sigma)$
							&	8.772217849		&	3.748318009			&	32.18853763		&	43.39896359	\\
\midrule
Erro acumulado em			&					&						&					&				\\
relação a distância (\%)	&	115\%			&	88\%				&	72\%			&	68\%		\\
\midrule
\bottomrule
\end{tabular}%
}{%
	\fonte{Produzido pelo autor.}%
}
\end{table}

A \autoref{table:ap-distance} mostra em sua última linha o erro acumulado
em relação a estimativa de distância (calculada a partir do valor de
potência médio), revelando um erro de tamanho assombroso.
Esse erro descarta qualquer possibilidade de associar a potência de sinal a uma
localização geográfica.

A definição do padrão IEEE 802.11 inclui que a potência de sinal
utilizada por uma estação para transmissão pode ser variada e, o critério utilizado
para a escolha desta potência varia entre fabricantes e não é padronizada.

\begin{citacao}[english]

	A STA may use any criteria, and in particular any path loss and link margin
	estimates, to dynamically adapt the transmit power for transmissions of an
	MPDU to another STA. The adaptation methods or criteria are beyond the scope
	of this standard. \

	\citeonline[10.8.6,p. 1047]{ieee80211}
\end{citacao}

Esta definição afasta ainda mais a possibilidade de associação entre a potência
de sinal recebida e a distância calulada através das equações de FSPL.

Em conclusão, esta seção mostra os níveis de erro encontrados
utilizando-se somente as equações de FSPL em um ambiente real e justifica o não
uso de valores RSS (potência de sinal) para determinar valores de geolocalização neste trabalho.

\section{Teste de localização com smartphone}
\label{sec:teste-smarphone}

Para verificar a capacidade do sensores de localizar contextualmente um
dispositivo móvel, smartphone, foi utilizado. Este foi posicionado em duas salas
diferentes, em cada uma das salas foi executada uma captura de 10 minutos. Para
que houvesse tráfego na rede o dispositivo móvel foi configurado para receber um
stream de vídeo no aplicativo \emph{Netflix}.

\begin{figure}[htb]
	\caption{\label{fig-planta-baixa}Ambiente de teste}
	\begin{center}
		\includegraphics[width=1\textwidth]{060-testes/data-analisis/planta-baixa-smartphone.png}
	\end{center}
	\legend{Fonte: Elaborada pelo autor}
	\nota[Em azul]{Sensores da aplicação}%
	\nota[Em vermelho]{Pontos do dispositivo teste}%
\end{figure}


No teste 1 o dispositivo estava na mesma sala do sensor 2, as distâncias
aproximadas em metros entre o ponto de teste e o sensor 1 é de \texttt{21,52m} e
entre o memso ponto e o sensor 2 é de \texttt{7,00m}. Foram capturados
\texttt{157 736} pacotes totalizando \texttt{9,7 MB} pelo sensor 1 e \texttt{21
974} pacotes totalizando \texttt{1.9 MB} de captura pelo sensor 2.

Para o teste 2 o dispositivo móvel estava posicionado no corredor fora da sala
do sensor 1 e distante do sensor 2, as distâncias aproximadas em metros entre o
ponto de teste e o sensor 1 é de \texttt{9,35m} e entre o memso ponto e o sensor
2 é de \texttt{20,14m}. Neste teste o sensor 1 capturou \texttt{103 555} pacotes
totalizando \texttt{6.4 MB}  de captura e o sensor 2 capturou \texttt{22 635}
pacotes totalizando \texttt{2 MB} de captura.

Posteriormente os arquivos de captura foram analisados com a ferramenta
\emph{Ron’s Editor} para que um sumário fosse construído.

\clearpage
\begin{figure}[ht]
	\centering
	\caption{\label{fig-mg4-noise-t1}Sumário de pacotes por dispositivo - Teste 1}
	\includegraphics[height=0.32\textheight,width=1\textwidth]{060-testes/data-analisis/distance-mg4plus-netflix/Teste1.png}
	\legend{A esquerda os pacotes recebidos pelo sensor 1 e a direita do sensor 2.
	Em preto o dispositivo de teste.
	Fonte: Elaborada pelo autor}
\end{figure}

\begin{figure}[hb]
	\centering
	\caption{\label{fig-mg4-noise-t2}Sumário de pacotes por dispositivo - Teste 2}
	\includegraphics[height=0.32\textheight,width=1\textwidth]{060-testes/data-analisis/distance-mg4plus-netflix/Teste2.png}
	\legend{A esquerda os pacotes recebidos pelo sensor 1 e a direita do sensor 2.
	Em preto o dispositivo de teste.
	Fonte: Elaborada pelo autor.}
\end{figure}

Pode-se obeservar que o \emph{smartphone} não representa a maioria dos pacotes
capturados, evidenciando a constatação do ambiente real e ruidozo em que o teste foi
executado.

\clearpage

Porém, uma vez filtrado os pacotes onde o endereço MAC do transmissor é o do
\emph{smartphone} podemos inferir os valores médios e o desvio padrão, chegando
nos seguinte valores.

\begin{table}[htb]
\IBGEtab{%
\ABNTEXchapterfont {
	\caption{\label{table:smartphone-distance}Análise dos pacotes do \emph{smartphone} - 10 minutos}%
}
}{%
\begin{tabular}{c|cc|cc}
\toprule
\midrule
Teste							& \multicolumn{2}{c|}{teste 1}			&	\multicolumn{2}{c}{teste 2}	\\
\midrule
Sensor 							&	Sensor 1		&	Sensor 2		&	Sensor 1		&	Sensor 2	\\
\midrule
Pacotes							&	33				&	25				&	21				&	9			\\
$\overline{dBm}$				&	-76.6969697		&	-49.88			&	-53.95238095	&	-68.55555556\\
$\sigma$						&	2.833778931		&	2.833137248		&	4.177034719		&	3.431876714	\\
\midrule
\midrule
$d(\overline{dBm})$				&	68.36730882		&	3.118889584		&	4.984470702		&	26.77797784	\\
$d(\overline{dBm} - \sigma)$	&	49.33550049		&	2.250832294		&	3.081536468		&	18.03781557	\\
$d(\overline{dBm} + \sigma)$	&	94.74088372		&	4.321722353		&	8.062519603		&	39.75315605	\\
Erro acumulado metros			&	45.40538323		&	2.070890059		&	4.980983136		&	21.71534048	\\
\midrule
\midrule
Distância real $M$				&	21.52			&	7				&	9.35			&	20.14		\\
$d(\overline{dBm}) - M$			&	46.84730882		&	3.881110416		&	4.365529298		&	6.637977837	\\
\midrule
\bottomrule
\end{tabular}%
}{%
	\fonte{Produzido pelo autor.}%
}
\end{table}

Os mesmos padrões encontrados na \autoref{sec:teste-ruido} aparecem: desvio
padrão grande, erro acumulado maior ainda.

Porém este trabalho limita-se a determinar o contexto (sala) e esta resposta pode
ser vista no contraste das \autoref{fig-mg4-t1} e \autoref{fig-mg4-t2} onde a
mudança do contexto é mais clara.

\begin{figure}[htb]
	\label{mg4-distance}
	\centering
	\begin{minipage}{0.49\textwidth}
	\centering
		\caption{\label{fig-mg4-t1}dBm Motorola G4+ - Teste 1}
		\includegraphics[width=1\textwidth]{060-testes/data-analisis/distance-mg4plus-netflix/target-Teste1.png}
		\legend{Fonte: Elaborada pelo autor}
	\end{minipage}
	\hfill
	\begin{minipage}{0.49\textwidth}
	\centering
		\caption{\label{fig-mg4-t2}dBm Motorola G4+ - Teste 2}
		\includegraphics[width=1\textwidth]{060-testes/data-analisis/distance-mg4plus-netflix/target-Teste2.png}
		\legend{Fonte: Elaborada pelo autor}
	\end{minipage}
\end{figure}


\chapter{Conclusão}
\label{chap:Conclusao}


Esse projeto teve como objetivo o aprofundamento na área de Internet das Coisas,
especialmente nas características de desenvolivmento local e independente que
foram encontradas durante a construção da aplicação localizadora de contexto de
dispositivos. Esta aplicação foi construida, instalada e testada no prédio
piloto fornecendo um ambiente real onde pode-se avaliar as reais capacidades de
um sistema localizador desse gênero.

Confirmou-se que um sistema de
geolocalização que utiliza somente FSPL com RSS tem poucas chances de ser
preciso. Porém, com as mesmas ferramentas, pode-se construir uma rede de sensores
onde cada nó é responsável por monitorar um contexto (sala, área ou parte de um
prédio) e neste sentido os dispositivos podem ser associados ao contexto e a
localização do sensor, efetivamente identificando a localização deles
e seus portadores dentro de um modelo lógico do prédio.

Para que a implementação pudesse ser replicada o custo associado foi determinado
como sendo de R\$ 324 por sensor onde é necessário um sensor por contexto além
de uma estrutura de rede já existente. Analogamente, o custo total do projeto
piloto foi definido como R\$ 995\footnote{Valor total das aquisições no
MercadoLivre.com} para os custos de \emph{harware} incluindo os protótipos e 400
horas\footnote{Estimado de 200 funcionalidades (\emph{git commit}) implementas
com média de 2 horas de implementação cada} de desenvolvimento.

Quanto ao estado da arte do ramo de Internet das Coisas foram identificadas
algumas características de destaque que nescecitam de nota:

\begin{alineas}
	\item Num ambiente onde
	idealmente todos os objetos e coisas estão conectados, poucos padrões globais
	existem para garantir essa conexão, tanto a nível jurídico (direitos e
	obrigações de fabricantes, desenvolvedores e usuários) quanto técnico
	(protocolos de comunicação, arquiteturas e recomendações unificados);

	\item Muitos protudos e soluções são muito jovens e carecem amadurecimento,
	especialmente no mercado residencial e comercial, onde as exigências de
	segurança, padronização, conformidade legal e disponibilidae são inferiores
	as encontradas no ramo industrial;

	\item Foi encontrado uma dificuldade durante a aquisição das plataformas pois,
	sem uma orientação de um profissional da área de sistemas embarcados, a
	comparação e escolha de uma plafarma é uma tarefa muito complexa;

	\item Comparado com a comunidade de desenvolvedores \emph{Web}, a comunidade
	de desenvolvedores IoT é muito jovem e não desenvolveu ferramentas para
	construir, compartilhar e reutiliazar projetos de maneira eficiente.
\end{alineas}


\section{Resultados para comunidade e trabalhos futuros}
\label{sec:trab-futuros}

Durante a exploração do tema foram encontradas diversas implementações de
localizadores baseados em Wi-Fi mas a implementação aqui executada mais se
assemelha com a de \citeonline{Ferreira2016} onde a mesma plataforma
(Raspberry Pi porém na sua versão 2 modelo B+), tipo de adaptador (Wi-Fi USB)
e \emph{software} (\emph{TShark}) foram utilizadas. A pricipal diferença são os
objetivos, enquanto a localização que \citeonline{Ferreira2016} buscou é do tipo
geografica, neste trabalho buscou-se o objetivo mais simples de encontrar o grau
de presença do dispositivo no mesmo contexto (sala) do sensor. Outras diferenças
são que algums desafios propostos por \citeonline{Ferreira2016} foram atacados
com certo nível de sucesso, entre eles: mais de um dispositivo sensor,
coleta e processamento simultâneos (\emph{online}), registro do histórico.
Outros que não renderam frutos também merecem atenção, como a exploração
adicional da plataforma ESP8266 que aqui propôem-se como trabalho futuro.


% ----------------------------------------------------------
% ELEMENTOS PÓS-TEXTUAIS
% ----------------------------------------------------------
\postextual
% ----------------------------------------------------------

% ----------------------------------------------------------
% Referências bibliográficas
% ----------------------------------------------------------
% \bibliography{900-referencias/referencias}
\bibliography{900-referencias/TCC-intro,900-referencias/TCC-problema,900-referencias/TCC-teorica,900-referencias/TCC-teorica-redes,900-referencias/TCC-teorica-iot,900-referencias/TCC-teorica-contexto,900-referencias/TCC-metodo,900-referencias/TCC-plataformas,900-referencias/TCC-construcao,900-referencias/TCC-conclusao,900-referencias/TCC-teorica-correlatos}

% ----------------------------------------------------------
% Glossário
% ----------------------------------------------------------
% Consulte o manual da classe abntex2 para orientações sobre o glossário.
\ifglossario
	Glossário
\fi

% ----------------------------------------------------------
% Apêndices
% ----------------------------------------------------------
\ifapendice
	% Inicia os apêndices
	\begin{apendicesenv}

	% Imprime uma página indicando o início dos apêndices
	\partapendices

	% ----------------------------------------------------------
	\chapter{Compras Mercado Livre}

	\includegraphics[scale=1.1,page=1]{920-apendices/Compras-MercadoLivre.pdf}
	\clearpage
	\includegraphics[scale=1.1,page=2]{920-apendices/Compras-MercadoLivre.pdf}
	\clearpage

	% ----------------------------------------------------------
	\end{apendicesenv}
\fi
% ---
\ifanexo
	Anexo de outros autores.
\fi

\ifindice
	Indice Remissivo
\fi

\end{document}
