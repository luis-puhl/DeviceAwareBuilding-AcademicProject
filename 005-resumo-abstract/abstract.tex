% resumo em inglês
\begin{resumo}[Abstract]
\begin{otherlanguage*}{english}

		IoT is at the focus of companies and enthusiasts alike because of its incredible
		growth with thousands of new devices every day, all built on top of the low
		processing costs (in both small hardware and large clouds) and the
		communication capability, that is increasingly required by
		busineesses and consumers, and present in daily things.

		Through the exploration of emerging platforms (such as ESP8266 and
		Raspberry Pi) and the construction of prototypes, this work aimed to
		construct a sensor that allows a building to contextually locate any
		device that communicates using Wi-Fi. Several
		technological tools including Raspberry Pi3, TShark, Node.js and MQTT
		have been used to reach this objective.

		These tools enabled tests where it was confirmed again that it is not
		possible to associate a geographic distance to received signal strength
		in the case of Wi-Fi communications, but with the same sensor it
		is possible to associate a device with the context of that sensor such as
		at a room inside a building.

	\textbf{Keywords}: Internet of Things. Raspberry Pi.
\end{otherlanguage*}
\end{resumo}
