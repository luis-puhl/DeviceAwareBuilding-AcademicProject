% resumo em português
\setlength{\absparsep}{18pt} % ajusta o espaçamento dos parágrafos do resumo
\begin{resumo}

	IoT é foco de empresas e entusiastas devido a seu incrível crescimento com
	milhares de novos dispositivos todos os dias, tudo isso construido sobre os
	baixos custos de processamento tanto em pequenos \emph{hardwares} quanto em
	grandes núvens e da capacide comunicacional que é cada vez mais exigida e
	presente em coisas do dia-a-dia.
	Através da exploração de plataformas emergentes (como o ESP8266 e Raspberry
	Pi) e da construção de protótipos este trabalho teve como objetivo
	construrir um sensor que permita que um prédio localize contextualmente (em
	qual sala) qualquer dispositivo que se comunique utilizando Wi-Fi. Para
	alcançar esse objetivo utilizou-se diversas ferramentas tencnologicas
	incluindo Raspberry Pi 3, TShark, Node.js e MQTT.
	Estas ferramentas possibilitaram testes onde confirmou-se novamente que não é
	possível associar uma distância geográfica a potência de sinal recebida
	(RSS) no caso de comunicações Wi-Fi porém, com o mesmo sensor é possível
	associar um dispositivo ao contexto de um sensor como a sala dentro de um
	prédio.

	\textbf{Palavras-chave}: Internet das Coisas, Raspberry Pi.
\end{resumo}
