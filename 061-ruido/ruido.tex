\section{Avaliação de ruído e consistência}
\label{sec:teste-ruido}

Para entender o ambiente onde a aplicação foi desenvolvida e testada no âmbito
de ruído e pontos de referência \emph{Wi-Fi}, foi executada uma captura de referência
durante a noite quando ninguém estava no prédio protótipo, ou seja, nenhum
dispositivo foi movimentado até a manhã seguinte.

Nesta captura, dois sensores foram posicionados a menos de 10
centímetros de distância um do outro sobre uma mesa a um metro do chão indicada no ponto
azul da \autoref{fig-planta-baixa-juntos}. A captura
ocorreu das \texttt{2:50} da noite até aproximadamente \texttt{11:25} da manhã, totalizando
aproximadamente \texttt{8} horas de captura.

\begin{figure}[htb]
	\caption{\label{fig-planta-baixa-juntos}Ambiente de teste de ruído}
	\begin{center}
		\includegraphics[width=1\textwidth]{060-testes/data-analisis/planta-baixa-ruido.png}
	\end{center}
	\nota[Em azul]{Sensores da aplicação}%
	\nota[Em verde]{Pontos de acesso da rede \emph{Wi-Fi} do laboratório}%
	\legend{Fonte: Elaborada pelo autor}
\end{figure}

As distâncias aproximadas, em linha reta, entre os sensores e o dispositivo
062722b3e5fe (AP na sala de servidores, anexo ao salão) é de
\texttt{14,6m} e entre os sensores e o dispositivo
062722b3e5fb (AP na sala de aula) é de \texttt{26,5m}.

Para o primeiro sensor, a função \emph{summary} da ferramenta \emph{Ron’s editor} indicou que foram capturados \texttt{1.729.624}
pacotes num arquivo de \texttt{155 MB} com \texttt{88} transmissores únicos.
Para o segundo sensor, a mesma função indicou que foram capturados \texttt{1.554.319}
pacotes num arquivo de \texttt{134 MB} com \texttt{66} transmissores únicos.
Em ambos os sensores, se destacaram dois endereços MAC que são os pontos de
acesso para rede \emph{Wi-Fi} do laboratório.

Os pacotes capturados dos pontos de acesso e os valores de potência de sinal
associados a eles são notáveis para entender a precisão de um sistema de localização
desta categoria uma vez que esses APs estão fixos e trasmitiram o
maior número de pacotes nesta captura.

Nas \autoref{fig-062722b3e5fb-s1}, \autoref{fig-062722b3e5fb-s2},
\autoref{fig-062722b3e5fe-s1} e \autoref{fig-062722b3e5fe-s2} podemos observar
a potência de sinal para cada pacote capturado em ordem de chegada. Na \autoref{fig-062722b3e5fb-s1} e
na \autoref{fig-062722b3e5fb-s2}, os pacotes
foram capturados, respectivamente, pelos sensores 1 e 2 para o ponto de acesso de MAC 062722b3e5fb. E na
\autoref{fig-062722b3e5fe-s1} e na \autoref{fig-062722b3e5fe-s2}, os pacotes foram capturados, respectivamente, pelos sensores 1 e 2 para
o ponto de acesso de MAC 062722b3e5fe.

\begin{figure}[htb]
	\begin{minipage}{0.49\textwidth}
	\centering
		\caption{\label{fig-062722b3e5fb-s1}Sinal em dBm por pacote capturado - 062722b3e5fb sensor 1}
		\includegraphics[width=1\textwidth]{060-testes/data-analisis/night-run/062722b3e5fb-sensor-01.png}
		\legend{Fonte: Elaborada pelo autor}
	\end{minipage}
\hfill
	\begin{minipage}{0.49\textwidth}
	\centering
		\caption{\label{fig-062722b3e5fb-s2}Sinal em dBm por pacote capturado - 062722b3e5fb sensor 2}
		\includegraphics[width=1\textwidth]{060-testes/data-analisis/night-run/062722b3e5fb-sensor-02.png}
		\legend{Fonte: Elaborada pelo autor}
	\end{minipage}
\hfill
	\begin{minipage}{0.49\textwidth}
	\centering
		\caption{\label{fig-062722b3e5fe-s1}Sinal em dBm por pacote capturado - 062722b3e5fe sensor 1}
		\includegraphics[width=1\textwidth]{060-testes/data-analisis/night-run/062722b3e5fe-sensor-01.png}
		\legend{Fonte: Elaborada pelo autor}
	\end{minipage}
\hfill
	\begin{minipage}{0.49\textwidth}
	\centering
		\caption{\label{fig-062722b3e5fe-s2}Sinal em dBm por pacote capturado - 062722b3e5fe sensor 2}
		\includegraphics[width=1\textwidth]{060-testes/data-analisis/night-run/062722b3e5fe-sensor-02.png}
		\legend{Fonte: Elaborada pelo autor}
	\end{minipage}
\end{figure}

Imediatamente, percebe-se que, apesar da distância ser invariável durate todo o
teste, a potência de sinal
não permaneceu constante. Nota-se também que o valor
nunca assume valor par. Estas duas características fazem com que o gráfico
seja feito de 7 linhas paralelas ao centro e alguns grupos de pontos ao redor.

Em todas os gráficos anteriores, nota-se uma abrupta mudança no comportamento próximo do pacote 225.000.
Estimou-se que esta mudança é devido ao horário (08:00) em que a universidade
(incluindo o prédio piloto e seus arredores) inicia o seu funcionamento.

Para ter uma visão geral clara da distribuição das potências de sinal, calculou-se
a média e o desvio padrão para cada um dos casos. Além disso, adicionou-se o
quanto o desvio padrão representa da média (erro).

\begin{table}[htb]
\IBGEtab{%
\ABNTEXchapterfont {
	\caption{\label{table:ap-pwr-avg}dBm Pontos de acesso - Acumulado 8 horas}%
}
}{%
\begin{tabular}{c|cc|cc}
\toprule
\midrule
AP							& \multicolumn{2}{c}{06:27:22:b3:e5:fe}		&	\multicolumn{2}{c}{06:27:22:b3:e5:fb}	\\
							&	Sensor 1		&	Sensor 2			&	Sensor 1		&	Sensor 2			\\
\midrule
Pacotes						&	300403			&	299864				&	305735			&	299264				\\
\midrule
Média (dBm)					& 	-57.66137822	&	-52.57644132		&	-73.04459417	&	-76.12900984		\\
\midrule
$\sigma$ (Desvio Padrão em dBm)	&	4.746974756		&	3.712407998			&	3.05045545		&	2.889560991			\\
$\sigma$ \%					&	8\%				&	7\%					&	4\%				&	4\%					\\
\midrule
%$3 \times \sigma $			&	14.24092427		&	11.13722399			&	9.151366351		&	8.668682972			\\
%$3 \times \sigma $ \%		&	25\%			&	21\%				&	13\%			&	11\%				\\
%\midrule
\bottomrule
\end{tabular}%
}{%
	\fonte{Produzido pelo autor.}%
}
\end{table}


Em alguns trabalhos, pode-se
encontrar a equação de FSPL (\emph{Free-space path loss} - perca no caminho em
espaço aberto) que é usualmente utilizada para determinar a potência do sinal
a ser transmitido para que este alcance o seu destinatário.

\begin{equation}
	{\mbox{FSPL(dB)}}=20\log _{{10}}(\text{d})+20\log _{{10}}(\text{f})+92.45
\end{equation}

Na aplicação \emph{Android} \emph{Wifi Distance Calculator}, desenvolvida por
\citeonline{Kuik2016}, e no trabalho de \citeonline{androidtri} a seguinte
equação é utilizada.

\begin{equation}
	\text{d} = 10 ^{ \frac{1}{20} \left( \text{p} - 20 \times log_{10}\left(\text{f}\right)  + 27.55 \right) } \\
\end{equation}

O $d$ é a distância em metros, $p$ é a potência do sinal em $dB$ e $f$
é a frequência do sinal em $MHz$. Como as redes \emph{Wi-Fi} utilizam canais na região
de 2.4GHz e 5GHz \cite{ieee80211}, pode-se simplificar a equação e chegar nos
resultados \autoref{eq:2.4ghz} e \autoref{eq:5ghz} respectivamente.

\begin{align}
\text{d}	&= 10 ^{ \frac{1}{20} \left( \text{p} - 20 \times log_{10}\left(\text{2400}\right)  + 27.55 \right) } \nonumber \\
			&= 10 ^{ \frac{1}{20} \left( \text{p} - 40.0542 \right) } \label{eq:2.4ghz}
\end{align}

\begin{align}
\text{d}	&= 10 ^{ \frac{1}{20} \left( \text{p} - 20 \times log_{10}\left(\text{5000}\right)  + 27.55 \right) } \nonumber \\
			&= 10 ^{ \frac{1}{20} \left( \text{p} - 46.4294 \right) } \label{eq:5ghz}
\end{align}

Utilizando a \autoref{eq:2.4ghz} (para 2.4GHz), pode-se inferir as distâncias entre os sensores
e pontos de acesso a partir da \autoref{table:ap-pwr-avg}. Na
\autoref{table:ap-distance}, estão presentes a distância em função da potência $d(p)$
calculada para a potência média ($\overline{dBm}$) e para o erro.


\begin{table}[htb]
\IBGEtab{%
\ABNTEXchapterfont {
	\caption{\label{table:ap-distance}Distância entre os sensores e os Pontos de acesso}%
}
}{%
\begin{tabular}{c|cc|cc}
\toprule
\midrule
AP							& \multicolumn{2}{c|}{06:27:22:b3:e5:fe}		&	\multicolumn{2}{c}{06:27:22:b3:e5:fb}	\\
							&	Sensor 1		&	Sensor 2			&	Sensor 1		&	Sensor 2			\\
\midrule
$d(\overline{dBm})$ metros
							&	7.639569932		&	4.254240777			&	44.89828049		&	64.03987741	\\
\midrule
$d(\overline{dBm} + \sigma)$ metros
							&	13.19525078		&	6.522926214			&	63.78998224		&	89.31585118	\\
$d(\overline{dBm} - \sigma)$ metros
							&	4.423032932		&	2.774608204			&	31.60144461		&	45.91688759	\\
\midrule
Erro acumulado metros		&					&						&					&				\\
$d(\overline{dBm}-\sigma)-d(\overline{dBm}+\sigma)$
							&	8.772217849		&	3.748318009			&	32.18853763		&	43.39896359	\\
\midrule
Erro acumulado em			&					&						&					&				\\
relação a distância (\%)	&	115\%			&	88\%				&	72\%			&	68\%		\\
\midrule
\bottomrule
\end{tabular}%
}{%
	\fonte{Produzido pelo autor.}%
}
\end{table}

A \autoref{table:ap-distance} mostra em sua última linha o erro acumulado
em relação a estimativa de distância (calculada a partir do valor de
potência médio), revelando um erro de tamanho assombroso.
Esse erro descarta qualquer possibilidade de associar a potência de sinal a uma
localização geográfica.

A definição do padrão IEEE 802.11 inclui que a potência de sinal
utilizada por uma estação para transmissão pode ser variada e, o critério utilizado
para a escolha desta potência varia entre fabricantes e não é padronizada.

\begin{citacao}[english]

	A STA may use any criteria, and in particular any path loss and link margin
	estimates, to dynamically adapt the transmit power for transmissions of an
	MPDU to another STA. The adaptation methods or criteria are beyond the scope
	of this standard. \

	\citeonline[10.8.6,p. 1047]{ieee80211}
\end{citacao}

Esta definição afasta ainda mais a possibilidade de associação entre a potência
de sinal recebida e a distância calulada através das equações de FSPL.

Em conclusão, esta seção mostra os níveis de erro encontrados
utilizando-se somente as equações de FSPL em um ambiente real e justifica o não
uso de valores RSS (potência de sinal) para determinar valores de geolocalização neste trabalho.
