\chapter{Conclusão}
\label{chap:Conclusao}

Durante a exploração do tema foram encontradas diversas implementações de
localizadores baseados em Wi-Fi mas a implementação aqui executada mais se
assemelha com a de \cite{Ferreira2016} onde a mesma plataforma
(Raspberry Pi porém na sua versão 2 modelo B+), tipo de adaptador (Wi-Fi USB)
e \emph{software} (\emph{TShark}) foram utilizadas. A pricipal diferença são os
objetivos, enquanto a localização que \citeonline{Ferreira2016} buscou é do tipo
geografica, neste trabalho buscou-se o objetivo mais simples de encontrar o grau
de presença do dispositivo no mesmo contexto (sala) do sensor. Outras diferenças
são que algums desafios propostos por \citeonline{Ferreira2016} foram atacados
com certo nível de sucesso, entre eles: mais de um dispositivo sensor,
coleta e processamento simultâneos (\emph{online}), registro do histórico.
Outros que não renderam frutos também merecem atenção, como a exploração
adicional da plataforma ESP8266 que aqui propôem-se como trabalho futuro.
