% ----------------------------------------------------------
% Inserir a ficha catalográfica
% ----------------------------------------------------------

% Isto é um exemplo de Ficha Catalográfica, ou "Dados internacionais de
% catalogação-na-publicação''. Você pode utilizar este modelo como referência.
% Porém, provavelmente a biblioteca da sua universidade lhe fornecerá um PDF
% com a ficha catalográfica definitiva após a defesa do trabalho. Quando estiver
% com o documento, salve-o como PDF no diretório do seu projeto e substitua todo
% o conteúdo de implementação deste arquivo pelo comando abaixo:
%
% \begin{fichacatalografica}
%	\includepdf{fig-ficha_catalografica.pdf}
% \end{fichacatalografica}

\ifficha
	\begin{fichacatalografica}
		\vspace*{\fill}					% Posição vertical
		\begin{center}					% Minipage Centralizado
			\fbox{
				\begin{minipage}[c][8cm]{13.5cm}		% Largura
					\hspace{0.5cm}
					\begin{minipage}{12.5cm}
						\small
						% \imprimirautor
						Puhl, Luís Henrique.
						%Sobrenome, Nome do autor

						\hspace{0.5cm} \imprimirtitulo \hspace*{1pt} / \imprimirautor, \imprimirdata

						\hspace{0.5cm} \pageref{LastPage} p. : il. \\

						\hspace{0.5cm} \imprimirorientadorRotulo~\imprimirorientador\\

						\hspace{0.5cm} \imprimirtipotrabalho ~--~
						Universidade Estadual Paulista. Faculdade de Ciências,
						\imprimirlocal, \imprimirdata\\

						\hspace{0.5cm} 1. Internet das Coisas.
						2. Raspberry Pi.
						3. Localização Contextual.
						4. MQTT.
						5. Node.js.
						6. TShark.
						7. Wi-Fi.
						I. Universidade Estadual Paulista "Júlio de Mesquita Filho". Faculdade de Ciências.
						II. Título
					\end{minipage}
					\hspace*{0.5cm}
				\end{minipage}
			}
		\end{center}
	\end{fichacatalografica}
	\vspace{2cm}
\fi
% ----------------------------------------------------------
