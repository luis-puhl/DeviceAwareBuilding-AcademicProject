%% abtex2-modelo-trabalho-academico.tex, laurocesar
%% Copyright 2012-2015 by abnTeX2 group at http://www.abntex.net.br/
%%
%% This work may be distributed and/or modified under the
%% conditions of the LaTeX Project Public License, either version 1.3
%% of this license or (at your option) any later version.
%% The latest version of this license is in
%%	http://www.latex-project.org/lppl.txt
%% and version 1.3 or later is part of all distributions of LaTeX
%% version 2005/12/01 or later.
%%
%% This work has the LPPL maintenance status `maintained'.
%%
%% The Current Maintainer of this work is the abnTeX2 team, led
%% by Lauro César Araujo. Further information are available on
%% http://www.abntex.net.br/
%%
%% This work consists of the files abntex2-modelo-trabalho-academico.tex,
%% abntex2-modelo-include-comandos and abntex2-modelo-references.bib
%%

% ------------------------------------------------------------------------
% ------------------------------------------------------------------------
% abnTeX2: Modelo de Trabalho Academico (tese de doutorado, dissertacao de
% mestrado e trabalhos monograficos em geral) em conformidade com
% ABNT NBR 14724:2011: Informacao e documentacao - Trabalhos academicos -
% Apresentacao
% ------------------------------------------------------------------------
% ------------------------------------------------------------------------

\documentclass[
	% -- opções da classe memoir --
	12pt,				% tamanho da fonte
	openright,			% capítulos começam em pág ímpar (insere página vazia caso preciso)
	oneside,			% para impressão em verso e anverso. Oposto a oneside
	a4paper,			% tamanho do papel.
	% -- opções da classe abntex2 --
	chapter=TITLE,		% títulos de capítulos convertidos em letras maiúsculas
	%section=TITLE,		% títulos de seções convertidos em letras maiúsculas
	%subsection=TITLE,	% títulos de subseções convertidos em letras maiúsculas
	%subsubsection=TITLE,% títulos de subsubseções convertidos em letras maiúsculas
	% -- opções do pacote babel --
	english,			% idioma adicional para hifenização
	french,				% idioma adicional para hifenização
	spanish,			% idioma adicional para hifenização
	brazil				% o último idioma é o principal do documento
	]{abntex2}

% Pacotes básicos
\usepackage{helvet}			% Usa a fonte Latin Modern - Mudei para Helvetica
\usepackage[T1]{fontenc}		% Selecao de codigos de fonte.
\usepackage[utf8]{inputenc}	% Codificacao do documento (conversão automática dos acentos)
\usepackage{lastpage}		% Usado pela Ficha catalográfica
\usepackage{indentfirst}		% Indenta o primeiro parágrafo de cada seção.
\usepackage{color}			% Controle das cores
\usepackage{graphicx}		% Inclusão de gráficos
\usepackage{microtype} 		% para melhorias de justificação


% Pacotes adicionais, usados apenas no âmbito do Modelo Canônico do abnteX2
\usepackage{lipsum}				% para geração de dummy text
\usepackage{customizacoes} 		% customizações feitas pelo autor

% Pacotes de citações
\usepackage[brazilian,hyperpageref]{backref}	% Paginas com as citações na bibl
\usepackage[alf]{abntex2cite}				% Citações padrão ABNT


% CONFIGURAÇÕES DE PACOTES
% Configurações do pacote backref
% Usado sem a opção hyperpageref de backref
\renewcommand{\backrefpagesname}{ }
% Texto padrão antes do número das páginas
\renewcommand{\backref}{\ABNTEXchapterfont}
% Define os textos da citação
\renewcommand*{\backrefalt}[4]{
	\ifcase #1 %
		%
	\or
		%
	\else
		%
	\fi}%

% Informações de dados para CAPA e FOLHA DE ROSTO
\titulo{Habilitando um Prédio a Localizar Contextualmente Dispositivos utilizando Redes Sem Fio}
\autor{Luís Henrique Puhl de Souza}
\local{Bauru}
\data{2016}
\orientador{Prof. Dr. Eduardo Martins Morgado}

\instituicao{%
  Universidade Estadual Paulista ``Júlio de Mesquita Filho''
  %\par
  Faculdade de Ciências - Campus Bauru
  %\par
  Departamento de Computação
}
\tipotrabalho{Monografia (Trabalho de Conclusão de Curso)}
% O preambulo deve conter o tipo do trabalho, o objetivo,
% o nome da instituição e a área de concentração
\preambulo{ Projeto de Trabalho de Conclusão de Curso de Bacharelado em Ciência
da Computação da Universidade Estadual Paulista ``Júlio de Mesquita Filho'',
Faculdade de Ciências, Campus Bauru }

% Configurações de projeto
\newif\iffinal
\finalfalse % define se é um arquivo final, se for não for retira umas partes.

\newif\ifabstract
\abstractfalse % define se mostra o abstract em inglês ou não.

% Configurações de aparência do PDF final
% alterando o aspecto da cor azul
\definecolor{blue}{RGB}{0,0,0}

% informações do PDF
\makeatletter
\hypersetup{
	  	%pagebackref=true,
		pdftitle={\@title},
		pdfauthor={\@author},
	 	pdfsubject={\imprimirpreambulo},
		 pdfcreator={LaTeX with abnTeX2},
		pdfkeywords={beacon}{raspberry pi}{internet das coisas}{abntex2}{trabalho acadêmico},
		colorlinks=true,		 		% false: boxed links; true: colored links
	 	linkcolor=blue,			 	% color of internal links
	 	citecolor=blue,		  		% color of links to bibliography
	 	filecolor=magenta,				% color of file links
		urlcolor=blue,
		bookmarksdepth=4
}
\makeatother

% Espaçamentos entre linhas e parágrafos
% O tamanho do parágrafo é dado por:
\setlength{\parindent}{1.3cm}

% Controle do espaçamento entre um parágrafo e outro:
\setlength{\parskip}{0.2cm}  % tente também \onelineskip

% compila o indice
\makeindex

% Início do documento
\begin{document}

% Seleciona o idioma do documento (conforme pacotes do babel)
%\selectlanguage{english}
\selectlanguage{brazil}

% Retira espaço extra obsoleto entre as frases.
\frenchspacing

% ----------------------------------------------------------
% ELEMENTOS PRÉ-TEXTUAIS
% ----------------------------------------------------------
\pretextual

\ABNTEXchapterfont {

% Capa
\imprimircapa

% Folha de rosto
% (o * indica que haverá a ficha bibliográfica)
%\imprimirfolhaderosto

% Inserir a ficha bibliografica

% Isto é um exemplo de Ficha Catalográfica, ou ``Dados internacionais de
% catalogação-na-publicação''. Você pode utilizar este modelo como referência.
% Porém, provavelmente a biblioteca da sua universidade lhe fornecerá um PDF
% com a ficha catalográfica definitiva após a defesa do trabalho. Quando estiver
% com o documento, salve-o como PDF no diretório do seu projeto e substitua todo
% o conteúdo de implementação deste arquivo pelo comando abaixo:
%
% \begin{fichacatalografica}
%	  \includepdf{fig_ficha_catalografica.pdf}
% \end{fichacatalografica}

\iffinal
  \begin{fichacatalografica}
	\sffamily
	\vspace*{\fill}					% Posição vertical
	\begin{center}					% Minipage Centralizado
	\fbox{\begin{minipage}[c][8cm]{13.5cm}		% Largura
	\small
	\imprimirautor
	%Sobrenome, Nome do autor

	\hspace{0.5cm} \imprimirtitulo  / \imprimirautor. --
	\imprimirlocal, \imprimirdata-

	\hspace{0.5cm} \pageref{LastPage} p. : il. (algumas color.) ; 30 cm.\\

	\hspace{0.5cm} \imprimirorientadorRotulo~\imprimirorientador\\

	\hspace{0.5cm}
	\parbox[t]{\textwidth}{\imprimirtipotrabalho~--~\\ \imprimirinstituicao,
	\imprimirdata.}\\

	\hspace{0.5cm}
		2. Internet das Coisas.
		I. \imprimirorientador.
		II. Universidade Estadual Paulista ``Júlio de Mesquita Filho''.
		III. Faculdade de Ciências.
		IV. Título
	\end{minipage}}
	\end{center}
  \end{fichacatalografica}
\fi

% INSERIR ERRATA
%\begin{errata}
%Elemento opcional da \citeonline[4.2.1.2]{NBR14724:2011}. Exemplo:

%\vspace{\onelineskip}

%FERRIGNO, C. R. A. \textbf{Tratamento de neoplasias ósseas apendiculares com
%reimplantação de enxerto ósseo autólogo autoclavado associado ao plasma
%rico em plaquetas}: estudo crítico na cirurgia de preservação de membro em
%cães. 2011. 128 f. Tese (Livre-Docência) - Faculdade de Medicina Veterinária e
%Zootecnia, Universidade de São Paulo, São Paulo, 2011.

%\begin{table}[htb]
%\center
%\footnotesize
%\begin{tabular}{|p{1.4cm}|p{1cm}|p{3cm}|p{3cm}|}
%  \hline
%	\textbf{Folha} & \textbf{Linha}  & \textbf{Onde se lê}  & \textbf{Leia-se}  \\
%	 \hline
%	 1 & 10 & auto-conclavo & autoconclavo\\
%	\hline
%\end{tabular}
%\end{table}

%\end{errata}


% INSERIR FOLHA DE APROVAÇÃO
% Isto é um exemplo de Folha de aprovação, elemento obrigatório da NBR
% 14724/2011 (seção 4.2.1.3). Você pode utilizar este modelo até a aprovação
% do trabalho. Após isso, substitua todo o conteúdo deste arquivo por uma
% imagem da página assinada pela banca com o comando abaixo:
%
% \includepdf{folhadeaprovacao_final.pdf}
%
\begin{folhadeaprovacao}
  \ABNTEXchapterfont {

	 \begin{center}

		{\ImprimirAutor}

		\vspace*{\fill}\vspace*{\fill}

		\begin{center}
		  \bfseries\large\ImprimirTitulo
		\end{center}

		\vspace*{\fill}

		\hspace{.45\textwidth}
		\begin{minipage}{.5\textwidth}
			 \imprimirpreambulo
		\end{minipage}%
		\vspace*{\fill}
	  \end{center}

	  %Trabalho aprovado. \imprimirlocal, 24 de novembro de 2012:

	  %\assinatura{\textbf{\imprimirorientador} \\ Orientador}
	  %\assinatura{\textbf{\imprimircoorientador} \\ Coorientador}
	  %\assinatura{\textbf{Professor} \\ Convidado 1}
	  %\assinatura{\textbf{Professor} \\ Convidado 2}
	  %\assinatura{\textbf{Professor} \\ Convidado 3}
	  %\assinatura{\textbf{Professor} \\ Convidado 4}
		\vspace*{0.5cm}
		\hspace{.5\textwidth}
	  \begin{center}
		 \ImprimirLocal \\ \imprimirdata
	  \end{center}
  }
\end{folhadeaprovacao}

% DEDICATÓRIA
\iffinal
  \begin{dedicatoria}
	\vspace*{\fill}
	\centering
	\noindent
	\textit{} \vspace*{\fill}
  \end{dedicatoria}
\fi

% AGRADECIMENTOS
\iffinal
	\begin{agradecimentos}
	\end{agradecimentos}
\fi

% EPÍGRAFE
\iffinal
  \begin{epigrafe}
	 \vspace*{\fill}
	\begin{flushright}
		\textit{}
	\end{flushright}
  \end{epigrafe}
\fi

% RESUMOS
\ifabstract
% RESUMO EM PORTUGUÊS
\setlength{\absparsep}{18pt} % ajusta o espaçamento dos parágrafos do resumo
\begin{resumo}
	\ABNTEXchapterfont {
		\textbf{Palavras-chave}: IoT. Internet das Coisas. GPS de interiores. Posicionamento Contextual.
	}
\end{resumo}

% RESUMO EM INGLÊS
\begin{resumo}[Abstract]
	\begin{otherlanguage*}{english}
		This is the english abstract.
		\vspace{\onelineskip}
		\noindent
		\textbf{Keywords}: IoT. Indoor GPS. Contextual Positioning.
	 \end{otherlanguage*}
\end{resumo}
\fi

% INSERIR LISTA DE ILUSTRAÇÕES
\iffinal
  \pdfbookmark[0]{\listfigurename}{lof}
  \listoffigures*
  \cleardoublepage
\fi

% INSERIR LISTA DE TABELAS
\iffinal
  \pdfbookmark[0]{\listtablename}{lot}
  \listoftables*
  \cleardoublepage
\fi

% INSERIR LISTA DE ABREVIATURAS E SIGLAS
\iffinal
  \begin{siglas}
	 \item[ANN] \textit{Artificial Neural Networks}
  \end{siglas}
\fi

% INSERIR LISTA DE SÍMBOLOS
\iffinal
  \begin{simbolos}
	 \item[$ \Gamma $] Letra grega Gama
	 \item[$ \Lambda $] Lambda
	 \item[$ \zeta $] Letra grega minúscula zeta
	 \item[$ \in $] Pertence
  \end{simbolos}
\fi

% INSERIR O SUMARIO
\pdfbookmark[0]{\contentsname}{toc}
\tableofcontents*
\cleardoublepage


% ------------------------------------------------------------------------------
% ELEMENTOS TEXTUAIS
% ------------------------------------------------------------------------------
\textual

\chapter[INTRODUÇÃO]{INTRODUÇÃO}
%\addcontentsline{toc}{chapter}{Introdução}

 1 - Introdução

Recentemente, Internet das Coisas (IoT - \textit{Internet of Things}) vem tomando o
foco das atenções de empresas e entusiastas de Tecnologia da Informação (IT -
\textit{Information Tecnology}) \cite{DzoneIoT:2015} a tal ponto que as empresas líderes do
segmento já incluem IoT como uma de suas áreas de atuação \cite{Ibm2016} \cite{ARM-mbed}
\cite{Microsoft2016} \cite{Intel2016} \cite{Oracle2016} \cite{Google2016} \cite{AmazonIoT2016}.

Todo este movimento no mercado é justificado pelo baixo custo dos pequenos
dispositivos computacionais \cite{RpiZeroLaunch} \cite{Esp8266.net} e grandes serviços na
nuvem \cite{Kaufmann2015} \cite{Amazon2016}. Este baixo custo possibilita a computação
ubíqua descrita por Weiser em 1991 e 1992 \cite{Weiser1999} que é entendida pelos
autores como \textit{``computação virtualmente onipresente''}. Também para os autores,
esta virtual onipresença é base e consequência para a IoT, sendo esta a
realizadora da computação ubíqua.

Uma vez contextualizado o mercado e a oportunidade de implementação da
computação ubíqua, percebemos a necessidade de dar aos elementos cotidianos
(coisas) a capacidade info-computacional, tornando-os sensores e atuadores
conectados, unicamente identificáveis e acessíveis através da rede mundial
(internet) \cite{Lemos2013} \cite{Kranenburg2012}.

É esperado que uma quantia total de 6,4 bilhões de dispositivos conectados
exista até o final de 2016 \cite{GARTNER2015} e entre 26 bilhões \cite{GARTNER2014} e 50
bilhões até 2020 com até 250 novas coisas conectando-se por segundo
\cite{CiscoBlog2013}.


--- texto antigo ---

Recentemente IoT (\textit{Internet of Things} - Internet das Coisas) vem tomando
o foco das atenções de empresas e entusiastas de IT (\textit{Information
Tecnology} - Tecnologia da Informação) \cite{DzoneIoT:2015} a tal ponto que as
empresas líderes do segmento já incluem IoT como uma de suas áreas de atuação
\cite{Ibm2016} \cite{ARM-mbed} \cite{Microsoft2016} \cite{Intel2016}
\cite{Oracle2016} \cite{Google2016} \cite{AmazonIoT2016}.

Todo este movimento no mercado é justificado pelo baixo custo dos pequenos
dispositivos computacionais \cite{RpiZeroLaunch} \cite{Esp8266.net} e grandes
serviços na nuvem \cite{Kaufmann2015} \cite{Amazon2016}. Este baixo custo
possibilita a computação ubíqua descrita por Weiser em 1991 e 1992
\cite{Weiser1999} que é entendida pelos autores como ``computação virtualmente
onipresente''. Também para os autores, esta virtual onipresença é base e
consequência para a IoT sendo esta a realizadora da computação ubíqua.

Uma vez contextualizado o mercado e a oportunidade de implementação da
computação ubíqua, percebemos a necessidade de dar aos elementos cotidianos
(coisas) a capacidade info-computacional, tornando-os sensores e atuadores
conectados, unicamente identificáveis e acessíveis através da rede mundial
(internet) \cite{Lemos2013} \cite{Kranenburg2012}.

\underline{Numero de dispositivos}

É esperado que uma quantia total de 6,4 bilhões de dispositivos conectados
exista até o final de 2016 \cite{GARTNER2015} e entre 26 bilhões
\cite{GARTNER2014}  e 50 bilhões até 2020 com até 250 novas coisas conectando-se
por segundo \cite{CiscoBlog2013}.


\chapter{PROBLEMA}
\label{chap:PROBLEMA}

 2 - Problema

Tamanha quantidade de dispositivos conectados pouco acrescenta na vida diária se
humanos ou coisas não puderem simplesmente se encontrar, tanto em ambiente real
quanto virtual é necessário o contato entre as partes para a existência de uma
interação.

Mais ainda, para melhor funcionamento de aplicações como o uso de conteúdo
específico para cada usuário e situação é necessário coletar informações
contextuais. Para a maioria das aplicações a informação contextual de maior
relevancia é a localização física.

Destacamos a necessidade da criação desta informação através de sensores ativos
sempre que necessário para que o dispositivo tenha ciência deste contexto em
suas tomadas de decisão e para que outros (sistemas, pessoas e coisas) saibam a
localização de qualquer dispositivo ao qual têm interesse de interagir.

Um exemplo da necessidade de localização de dispositivos dentro de um prédio
seria um profissional saber onde está o dispositivo em seu local de trabalho,
seja ele um vendedor e seu tablet para demostrar um produto fora de estoque em
uma loja ou um médico e um desfibrilador.

--- texto antigo ---

\textbf{\underline{Problema}}
\underline{Grande quantidade de dispositivos}

Tamanha quantidade de dispositivos conectados pouco acrescenta na vida diária se
humanos ou coisas não puderem simplesmente se encontrar, tanto em ambiente real
quanto virtual é necessário o contato entre as partes para a existencia de uma
interação.

Mais ainda, para melhor funcionamento de aplicações, em especial o oferecimento
de conteúdo específico para cada usuário, é necessário contextualizar e o
primeiro passo da contextualização é a conciência da localização.

\underline{Exemplo da perda de aparelhos no predio (zebra) (ou outro exemplo)}

Um exeplo da necessidade de localização de dispositivos dentro de um prédio
seria um profissional saber onde está o dispositivo em seu local de trabalho,
seja ele um vendedor e seu tablet para demostrar um produto fora de estoque em
uma loja ou um médico e um desfibrilador.

A grande quantidade de dispositivos traz o desafio de localizá-los
contextualmente, tanto para que o dispositivo tome ciência de sua posição em um
contexto além de sua posição global em suas tomadas de decisão e para que outros
(sistemas, pessoas e coisas) saibam a localização de qualquer dispositivo ao
qual têm interesse de interagir.

Mesmo com a grande quantidade de dispositivos já conectados são poucos os
documentos descrevendo boas práticas para concepção, construção e manutenção de
aplicações IoT, especialmente sobre os cuidados tomados quanto a segurança e
análise de custos para a implementação e manutenção.

Além disso, a falta de referências neste sentido é agravada quando considera-se
implementação no interior do estado de São Paulo visto que poucas são as
organizações atualizadas neste tema levando a uma falta enorme de conteúdo
escrito na linguagem local além de serviços e produtos disponíveis para
construção de uma plataforma completa e competitiva nesta região.


\section{SOBRE SISTEMAS DE POSICIONAMENTO}
\label{sec:SOBRE SISTEMAS DE POSICIONAMENTO}

 2.1 -

Sistemas de posicionamento (PS - \textit{Positioning System}) são geralmente
constituidos de um Ponto Origem Global escolhido (\textit{O}) e um conjuto não vazio de
Pontos de Referência (RP - \textit{Reference Point}) cuja localização global em relação
ao \textit{O} é conhecida com precisão maior ou igual a oferecida pelo sistema.

Também faz parte do sistema o ponto móvel (MU - \textit{Mobile User}) sobre o qual
temos interesse em determinar a posição que é feita pelo PS encontrando uma
distância (com dimensão variável de acordo com o método utilizado para
adiquiri-la) relativa a um sub-conjunto de RPs. Feito isso é possível utlizar
modelos matemáticos para a partir das distâncias encontrar uma posição do MU em
relação aos RPs e uma nova transformação é aplicada para encontrar a posição
relativa ao \textit{O}.

Uma das maneiras de classificar PSs é entre os de Auto Posicionamento e
Posicionamento Remoto. Os de Auto Posicionamento contém no MU todo aparato
necessário para medir a distância dos RPs e calcular a posição em relação a \textit{O}.
Já os de Posicionamento Remoto tem o mínimo necessário na MU e todo o trabalho
de cálculo de distância e posição global é feito nos RPs ou em uma unidade
coordenadora destes.

Para PSs eletônicos baseados em radio-frequência (RF - \textit{Radio Frequency})
geralmente utilizam-se dois componetes básicos, Transmissores e Receptores, os
quais assume-se que ao menos um destes está no RP e ao menos um outro no MU.
Para calcular a distância entre MU e RP utiliza-se as propriedades da
comunicação por RF como tempo de chegada (TOA - \textit{Time Of Arrival}), diferenial
de tempo de chegada (TDOA - \textit{Time Difference Of Arrival}) e ângulo de chegada de
sinal (AOA - \textit{Angle Of Arrival}).

Para maior precisão, é comum a utilização de múltiplas RPs geralmente com o
número mínimo igual ao número de dimenções espaciais que deseja-se calcular.
Notamos que para sistemas distribuídos a sincronização de relógios é um
problema clássico então o tempo conta como dimensão.

Os sistemas classificados como ``Sistema de Navegação Global por Satélite'' (GNSS -
\textit{Global Navigation Satellite System}), como o tradicional Estadunidense Sistema
de Posicionamento Global (GPS - \textit{Global Positioning System}), utilizam a técnica
em que o dispositivo móvel contém o receptor e os transmissores são fixos em
satélites na órbita terrestre \cite{Djuknic2001}. Devido a posição e número de
satélites, o GPS e seus correlatos estão sempre presentes do ponto de vista de
um observador da superfície terrestre, sendo para este tipo de usuário um
sistema ubíquo.

Entretanto a força do sinal GNSS não é suficiente para penetrar a maioria dos
prédios, uma vez que estes dependem de visão direta (LOS - *Line-Of-Sight*)
entre os satélites e o receptor. A reflexão do sinal muitas vezes permite a
leitura em ambientes fechados, porém o cálculo da posição não será confiável
\cite{Dartmouth2000}.Portanto, apesar da ubiquidade dos GNSSs em ambientes abertos,
são necessárias soluções diferentes para obter um Sistema de Posicionamento para
Ambientes Fechados (IPS - \textit{Indoor Positioning System}) sendo a ubiquidade deste
essencial para conquistar o mesmo nível de confiança trazido pelos GNSSs.

Para implementar este IPS, propomos o uso de tecnologias já implantadas em
dispositivos móveis e essenciais para o funcionamento dos memos, especialmente
as de camadas de comunicação, que são ubíquoas no ambiente dos dispositivos
móveis, como Wi-Fi (padrão *IEEE 802.11*) e Bluetooth (padrão \textit{Bluetooth SIG}),
para que os objetos conectados em que temos interesse de encontrar o contexto
locativo não necessitem de modificações.

---

\underline{Soluções corelatas}

\underline{Nota do autor: Introduza os meios existentes de localização (marcelo).}

% GPS não funciona bem indoor, existem 2 tipos de indoor

Os tradicionais sistemas de GPS (Global Positioning System) utilizam a técnica
de auto posicionamento para calcular sua posição no globo terrestre baseado nos
sinais recebidos de 24 satélites posicionados na órbita terrestre com 20.200
kilômetros de distância entre cada um \cite{Djuknic2001}.


Entretanto, a força do sinal GPS não é suficiente para penetrar a maioria dos
prédios. A reflexão do sinal muitas vezes permite a leitura em ambientes
fechados, porém o cálculo da posição não será confiável
\cite{Dartmouth2000}. Portanto, são necessárias soluções diferentes para
se criar um sistema de geoposicionamento que funcione em ambientes fechados.

Por exemplo, utilizando uma série de sensores wi-fi posicionados em pontos fixos
dentro de um prédio, com a triangulação do sinal é possível calcular a posição
de dispositivos conectados à rede wi-fi [BLECKY, 2016].

Para oferecer uma posição confiável, é necessário que estes sensores coletem e
transmitam a força do sinal wi-fi em cada dispositivo com uma alta frequência.


\chapter{JUSTIFICATIVA}
\label{chap:JUSTIFICATIVA}

 3 - Justificativa

Sobre o contexto encontrado, propomos um ambiente consciente onde o contexto
locativo oriundo do posicionamento remoto de cada dispositivo móvel é
administrado e divulgado pelo prédio conectado ao invés da auto localização do
aparelho, pois:

- Uma vez encontrada a localização, é mais fácil propagar esta informação do
ambiente para o aparelho em comparação ao auto posicionamento, pois a negociação
entre o ambiente e o aparelho é nula quando o primeiro contém a informação- o
ambiente sempre disponibilizará uma informação coletada para o gerador desta
informação;
- Pode-se lidar com grande heterogeniedade de dispositivos, uma vez
que cada um deles não precisa se adaptar para cada mudança de ambiente;
- Este tipo de informação já é contida nos históricos de cada Ponto de Acesso
 Wi-Fi (AP - \textit{Access Point}), porém:
	- Geralmente sem uso - poucas são as aplicações que usam a localização
	obtida pelo AP;
	- com granularidade insuficiente para uso em aplicações contextualizadas;
	- geralmente não disponibilizada pelos APs.
- Uma vez instalado um PS deste gênero, a quatia de dispositivos que ele pode
localizar fica limitada apenas pela rede física;
- Economia de hardware quando menos é exigido de cada dispositivo;

Levamos em conta também a quantidade prevista de em média 5 dispositivos IoT
por pessoa que seriam beneficiados sempre que utilizados no ambiente conectado.

![Modelo das camadas][projeto]
Fonte: Marcelo Augusto Cordeiro

[projeto]: latex/img/projeto.JPG

A Figura \ref{fig:projeto} apresenta a arquitetura simplificada de uma aplicação
IoT.

Para possibilitar testes em um ambiente real, o projeto aqui proposto será
instalado dentro do prédio do Laboratório de Tecnologia da Informação Aplicada
(LTIA) da Faculdade de Ciências da Unesp de Bauru.


--- texto antigo ---

\underline{Economia em cada aparelho}\\

Escolhemos o sistema de antenas sensoras e dispositivos transmissores ao invés
do contrario (explicar com o detalhes o pq da escolha e falar o que é contrário)
para economia de harware no sentido de menos hardware em cada dispositivo e
levando em conta  a quantidade prevista de em média 5 dispositivos conectados
por pessoa e de que, com este sistema, estes dispositivos não precisariam de
sensores para localizar-se além de informações mais completas para o prédio.

\underline{Funcionamento por prédio} \\
 -> \underline{uma vez instalado qualquer quantidade de
devices é recebida, facil gerencia dentro do predio}\\
 -> \underline {beneficios para o admin do predio}\\


Utilizando como exemplo o prédio do Laboratório de Tecnologia da Informação
Aplicada (LTIA) da Faculdade de Ciências da Unesp de Bauru, em um dia comum, é
observado uma média de 30 dispositivos conectados à rede wi-fi.

Considerando um sensor que a cada 30 segundos colete 1 kB de dados de cada
dispositivo, por mês, seriam coletados mais de 2 GB de dados. Portanto, para
garantir um sistema escalável, é necessário a utilização de técnicas de Big Data
para armazenar e manipular esses dados.

O melhor modo de se definir Big Data ainda é discutido por pesquisadores, mas
uma definição simples é a de que “se é necessário se preocupar com o tamanho dos
dados, então é Big Data.” (ESPOSITO, 2015, tradução nossa)

\underline{Nota do parecerista: Reescrever}

Na visão dos autores, promover o desenvolvimento local através de trabalhos
exemplo, treinamentos ou manuais é fundamental para a equiparação dos
desenvolvedores locais com as tecnologias e tendencias de mercado então
justificamos sua execução para que outras organizações possam encontrar novos
caminhos.

\chapter{OBJETIVOS}
\label{chap:OBJETIVOS}

 4 - Objetivos

\section{OBJETIVO GERAL}
\label{sec:OBJETIVO GERAL}

 4.1 - Objetivos Gerais

Considerando características locais, propõem-se a construção de uma aplicação
para localizar contextualmente dispositivos dentro de um prédio piloto e avaliar
sua precisão.

Além desta aplicação, é objetivo definir o custo do projeto piloto, incluindo
esforço de pesquisa assim como definir um custo para replicação deste
localizador contextual em outros prédios.

\section{OBJETIVOS ESPECÍFICOS}
\label{sec:OBJETIVOS ESPECÍFICOS}

 4.2 Objetivos Específicos

-  Estabelecer o estado da arte sobre a desenvolvimento de aplicações IoT;
-  Identificar desafios locais para o desenvolvimento;
-  Identificar provedores de serviços, dispositivos e ferramentas para o
desenvolvimento;
- Construção de sensores de identificação e localização (distância) de
 dispositivos cuja comunicação seja baseada em Bluetooth e Wi-Fi;
- Posicionamento destes sensores;
- Construção de um dispositivo agregador de informações dos sensores
 (\textit{gateway}) e sua interface web (Web REST API - *Representational
State Transfer Application Programming Interface*);
-  Estimar o custo total do projeto piloto incluindo esforço de pesquisa;
-  Estimar o custo de replicação da aplicação em outros prédios.

--- texto antigo ---

\textit{Objetivos}

\textit{Objetivos Gerais}

Considerando características locais, propõem-se a construção de uma aplicação
para localizar contextualmente dispositivos dentro de um prédio piloto e avaliar
sua precisão.

Além desta aplicação, é objetivo definir o custo do projeto piloto incluindo
esforço de pesquisa assim como definir um custo para replicação deste
localizador contextual em outros prédios.

\textit{Objetivos Específicos}

\begin{alineas}

	\item Estabelecer o estado da arte sobre a desenvolvimento de aplicações IoT;

	\item Identificar desafios locais para o desenvolvimento;

	\item Identificar provedores de serviços, dispositivos e ferramentas para o
	desenvolvimento;

	\item Construir um protótipo de sala conectada virtualmente que identifique
	os dispositivos conectados a rede que existem dentro nela através de
	conexões sem fio;

	\item Estimar o custo total do projeto piloto incluindo esforço de pesquisa;

	\item Estimar o custo de replicação da aplicação em outros prédios.

\end{alineas}

\underline{Nota do parecerista: Na proposta do Marcelo, foi informado que o Luís
Henrique ficará responsável por todos os sensores, o que inclui sua construção,
posicionamento, configuração e programação dos gateways.}


\chapter{MÉTODO DE PESQUISA}
\label{chap:MÉTODO DE PESQUISA}

 5 - Método de Pesquisa

Abordagens para medir distâncias através de redes sem fio Wi-Fi
(\cite{bahillo2009ieee}) e Bluetooth já existem e, propor novas maneiras não é o foco
deste trabalho. Utilizando essas técnicas, propomos estabelecer uma rede de nós
sensores colaborativos fixos no ambiente onde deseja-se obter a localização dos
dispositivos. As informações de distância serão compartilhadas entre os nós para
maior precisão da informação.

Para a implementação pretende-se utilizar os softwares de maior destaque
recentemente nos ramos de comunicação de baixa energia (\textit{MQTT}), serviços \textit{Web}
para armazenamento (\textit{MongoDB}) e publicação (\textit{NodeJS}), além de softwares para
medição da distância sem interferir na comuncação (\textit{Sniffing}) e das plataformas
de hardware disponíveis e recomendadas para IoT com capacidade Wi-Fi e Bluetooth
(\textit{Raspberry Pi 3}).

Mesmo com a grande quantidade de dispositivos já conectados são poucos os
documentos descrevendo boas práticas para concepção, construção e manutenção de
aplicações IoT, especialmente sobre os cuidados tomados quanto a segurança e
análise de custos para a implementação e manutenção.

Além disso, a falta de referências neste sentido é agravada quando considera-se
a implementação no interior do estado de São Paulo. Nesta região, poucas são as
organizações atualizadas neste tema, levando a uma falta enorme de conteúdo
escrito na linguagem local além de serviços e produtos disponíveis para
construção de uma plataforma completa e competitiva nesta região.

Devido a falta de conteúdo e instrução, utilizaremos prototipagem ágil para este
projeto, uma vez que esta metodologia de desenvolvimento é recomendada para
projetos cujas especificações e definições não são claras, demandando muitas
modificações das mesmas durante a execução do mesmo. Esse método entra em
contraste com metodologias clássicas, como a cascata que apesar de previsíveis,
não reagem bem a ambientes de extrema incerteza.

Mais especificamente utilizaremos uma variante da metodologia \textit{Scrum}
\cite{James2016} que será adaptada para o projeto. Nela, serão executadas iterações
de uma semana em que a cada iteração, uma nova versão melhorada do produto
completo (hardware, software, documentação e resultados) será entregue.

Dentro de cada iteração, as camadas da aplicação IoT serão escolhidas,
implementadas, justificadas e avaliadas, sendo todo processo documentado. Como
resultado de cada uma delas, será gerado um relatório das mudanças a partir da
iteração anterior.

Com mais detalhes, cada iteração cumprirá uma parte de cada objetivo no trabalho
completo levando o projeto integralmente para um estágio de completude maior a
cada iteração. Serão foco de cada iteração os objetivos abaixo, gerando um
relatório utilizado para tomar e justificar decisões durante a execução do
projeto bem como servir de posterior documentação. Os objetivos de cada iteração
são:

- Escolha de provedores de serviços, dispositivos e ferramentas para o
desenvolvimento;
- Construção, avaliação, teste e manutenção dos sensores;
- Construção de um dispositivo agregador e sua API;
- Estimativa do custo total do projeto piloto;
- Estimativa do custo de replicação;
- Identificação dos desafios para o desenvolvimento.

Desta forma, esperamos garantir a liberdade necessária para o projeto ser
executado com sucesso, mesmo no ambiente de incerteza no qual o mercado local de
IoT encontra-se, cumprindo as premissas de de funcionamento, mantebilidade e
segurança que são grande importância para os interessados na área.


--- texto antigo ---

\textit{Método de Pesquisa}

Utilizaremos prototipagem ágil semelhante ao desenvolvimento de um produto
utilizando a metodologia \textit{Scrum} \cite{James2016}, executando iterações
de uma semana onde a cada iteração uma nova versão melhorada do produto completo
(hardware, software, documentação e resultados) será entregue.

Dentro de cada iteração as camadas da aplicação IoT serão escolhidas,
implementadas, justificadas e avaliadas sendo todo processo documentado. Como
resultado de cada iteração será gerado um relatório das mudanças a partir da
iteração anterior.

\begin{figure}[htb]
	\caption{\label{fig:projeto}Modelo das camadas }
	\begin{center}
		\includegraphics[width=1\textwidth]{img/projeto.JPG}
	\end{center}
	\legend{Fonte: Marcelo Augusto Cordeiro}
\end{figure}

A Figura \ref{fig:projeto} apresenta a arquitetura simplificada de uma aplicação
IoT. Esta será modificada a cada iteração do projeto especialmente as camadas de
sensores, \textit{gateway} e base de dados.

\underline{Nota do parecerista: Reescrever a metodologia....descrever
detalhadamente as atividades que deverão ser desenvolvidas}


\chapter{CRONOGRAMA}
\label{chap:CRONOGRAMA}


5 - Cronograma

Devido a natureza ágil e iterativa da metodologia, o cronograma será dividido em
apenas três partes: Levantamento Bibliográfico Inicial, Desenvolvimento
Iterativo (Escolha de provedores e fornecedores; Construção, avaliação, teste e
manutenção dos sensores e agregadores; Estimativas de custos totais e de
replicação e Documentação de desenvolvimento) e Revisão Final. Estas partes
serão distribuídas conforme a Tabela 1.

Tabela 1 – Cronograma de Atividades Propostas

| Atividade															 	| 	Fev	 | 	Mar	 | 	Abr	 | 	Mai	 | 	Jun	 | 	Jul	 | 	Ago	 | 	Set	 | 	Out  |
| --------------------------------------------------------------------- |:------:|:-----:|:-----:|:-----:|:-----:|:-----:|:-----:|:-----:|:-----:|
| Levantamento Bibliográfico Inicial									| 	X	 | 	X	 | 	 	 | 	 	 | 	 	 | 	 	 | 	 	 | 	 	 | 	     |
| Escolha de provedores e fornecedores									| 	 	 | 	X	 | 	X	 | 	X	 | 	X	 | 	X	 | 		 | 		 | 	     |
| Construção, avaliação, teste e manutenção dos sensores e agregadoes	| 	 	 | 		 | 	X	 | 	X	 | 	X	 | 	X	 | 	X	 | 		 | 	     |
| Estimativas de custos totais e de replicação							| 	 	 | 		 | 		 | 		 | 	X	 | 	X	 | 	X	 | 	X	 | 	     |
| Documentação de desenvolvimento										| 	 	 | 		 | 	X	 | 	X	 | 	X	 | 	X	 | 	X	 | 	X	 | 	     |
| Revisão Final															| 	 	 | 	 	 | 	 	 | 	 	 | 	 	 | 	 	 | 	x 	 | 	X	 | 	X    |
Fonte: Produzido pelo autor

--- texto antigo ---

\textit{Cronograma}


Devido a natureza ágil e iterativa da metodologia, o cronograma será dividido em
apenas três partes: Levantamento Bibliográfico Inicial, Desenvolvimento
Iterativo e Revisão Final. Estas partes serão distribuídas conforme a
Tabela~\ref{table:cronograma}.

\begin{table}[htb]
\IBGEtab{%
\ABNTEXchapterfont {
  \caption{Cronograma de Atividades Propostas}%
  \label{table:cronograma}
}
}{%
  \begin{tabular}{cccccccccc}
  \toprule
	Atividade							&	Fev	&	Mar	&	Abr	&	Mai	&	Jun	&	Jul	&	Ago	&	Set	&	Out \\
  \midrule \midrule
	Levantamento Bibliográfico Inicial	&	X	&	X	&	 	&	 	&	 	&	 	&	 	&	 	&	  \\
  \midrule
  Desenvolvimento Iterativo				&	 	&	X	&	X	&	X	&	X	&	X	&	X	&	X	&	  \\
  \midrule
  Revisão Final							&	 	&	 	&	 	&	 	&	 	&	 	&	 	&	X	&	X \\
  \bottomrule
\end{tabular}%
}{%
  \fonte{Produzido pelo autor.}%
  }
\end{table}

\underline{Nota do parecerista: A atividade ``Desenvolvimento Iterativo'' deve ser dividida em sub-atividades...}


% ----------------------------------------------------------------------------

% ELEMENTOS PÓS-TEXTUAIS


% Referências bibliográficas
\bibliography{referencias}


\end{document}
